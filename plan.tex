%!TEX TS-program = xelatex
%!TEX encoding = UTF-8 Unicode

\def \papersize {a4paper}
\documentclass[12pt,\papersize]{extarticle}
% extarticle is like article but can handle 8pt, 9pt, 10pt, 11pt, 12pt, 14pt, 17pt, and 20pt text

\def \ititle {Mindreading and Joint Action: Philosophical Tools}
\def \isubtitle {Notes for lectures, CEU Fall 2012-13}
\def \iauthor {Stephen A. Butterfill}
\def \iemail{s.butterfill@warwick.ac.uk}
%for anonymous submisison
%\def \iauthor {}
%\def \iemail{}
%\date{}

\input{$HOME/Documents/submissions/preamble_steve_paper2}

%avoid overhang
\tolerance=5000


\begin{document}

\setlength\footnotesep{1em}

\bibliographystyle{newapa} %apalike

%these two lines are for anonymous submission --- they remove author and date
%but don't forget to remove defs above as well --- otherwise it will be in the metadata
%\author{}
%\date{}


\maketitle
%\tableofcontents

\begin{abstract}
\noindent
***
\ 

\noindent
\textbf{Keywords:}
Mindreading, Joint Action, Action, Belief, Intention, Representation, Mental State
\end{abstract}




\section{Mindreading and Tracking}

\subsection{Mindreading}
Mindreading is 
	the process of 
	identifying mental states and actions 
	as the mental states and actions 	of a particular subject 
	on the basis, ultimately, of bodily movements and their absence,
somewhat as reading is the process of identifying propositions on the basis of inscriptions \citep[p.\ 4]{Apperly:2010kx}.

%Mindreading is the representation of mental states as the mental states of a particular subject.


\subsection{Tracking mental states}
It is useful to contrast mindreading with tracking.

An ability to \textit{track} mental states of a particular kind (e.g. beliefs) is ability that exists in part because exercising it brings benefits obtaining which depends on exploiting or influencing facts about others's mental states.

To track beliefs (say) is to exercise a belief-tracking ability.

To illustrate, suppose that Hannah is able to discern whether another's eyes are in view, that Hannah exercises this ability to escape detection while stealing from others, that Hannah's ability exists in part because it benefits her in this way, and that Hannah's escaping detection depends on exploiting a fact about other's visual representations (namely that they usually cannot see Hannah's acts of theft when Hannah doesn't have their eyes in view).
Then Hannah has an ability to track visual representatios. 
(This is not supposed to be a plausible, real-world example but only to illustrate what the definition requires.)

Sometimes tracking involves mindreading, but not always (as the story about Hannah and the eyes illustrates).

Tracking mental states without mindreading is probably quite common.  Another example: preening (?).


\section{That we do not understand what mindreading is}
Empirical questions about mindreading include:
\begin{itemize}
\item When in development does mindreading first occur?
\item What representations and processes make mindreading possible?
\item Is mindreading automatic?
\item Which animals are capable of mindreading?
\end{itemize}
%
Much progress has been made on these questions, and there is more still to make. 
I want to suggest that there is also an obstacle to progress.
The obstacle is that we don't adequately understand what mindreading is. 

Why think that we don't adequately understand what mindreading is? 
The strongest reason is this.
Some apparently puzzling patterns in findings about mindreading can be resolved by thinking carefully about what mindreading is. 
But we'll only be in a position to evaluate this claim right at the end, when we have reflected on what mindreading is.

There are, though, some hints that we might not adequately understand what mindreading is.
As we'll see, there are controversies concerning what mental states are, and what actions are.  
But when the topic is mindreading, these controversies are usually ignored and it is assumed that we all know what actions and mental states are. 
To better understand what mindreading is we will need to reflect on what actions and mental states are.

So my plan is to step back from empirical questions about mindreading and first focus on more narrowly philosophical issues about what mindreading is.
Having done this, we'll come back to the empirical questions about mindreading.  
The philosophical part is valuable to the extent that it supports progress with questions about when, how and where mindreading occurs.

But you might still be sceptical that philosophy is really needed.  Do we really not adequately understand what mindreading is?
You probably shouldn't take my word for it.
After all, not understanding things is what I do for a living.
So consider these questions:
\begin{itemize}
%
\item What evidence could in principle support the ascription of a particular belief to a given subject, and how does the evidence support the ascription?
%
\item Could a mindreader be able to identify beliefs despite not  understanding what it is for a belief to be true or false? 
%
\item Does being a mindreader entail being able, sometimes, to identify one's own mental states and actions? 
%
\item Could there be mindreaders who can identify intentions and knowledge states but not beliefs?
%
\item Does identifying an action necessarily involve representing an intention?
%
\end{itemize}
If we fully understood what mindreading was, we would be able to answer these questions in a principled way.
The fact that we can't shows that we don't fully understand what mindreading is.
And it suggests that we don't adequately understand it either.

To better understand what mindreading is we have to take a step back and ask what actions are and what mental states are.



%[NB: postpone empirical puzzles about mindreading until towards the end; use these to illustrate applications of the ideas.]




\section{What are mental states?}

mental state = 
	\\ \hspace*{10 mm} subject [e.g. Ayesah] 
	\\ \hspace*{10 mm} + 
	\\ \hspace*{10 mm} attitude [e.g. desire] 
	\\ \hspace*{10 mm} + 
	\\ \hspace*{10 mm} content [e.g. that Ayesha eats ice cream]

The subject is just an object.

Explain attitude and content using 2 x 2: 




\begin{table}[htbp]


\begin{center}
\footnotesize	%shrink for better spacing
\begin{tabular*}{1\textwidth}{@{\extracolsep{\fill}} l c *{3}{cc} } 

\toprule
& \multicolumn{3}{c}{attitude} 
\\ \cmidrule(r){2-4}
%
 & belief & desire & ...
%
\\ \midrule
%
Ayesha eats ice cream & 1 & 3 & ...
\\
Frederique writes poetry & 2 &  5 & ...
\\
... & ... & ... & ...
\\
%
\bottomrule
%
\end{tabular*}
\caption{Attitude versus content}
\end{center}	%careful -- position of this affects distance between table and caption(!)
\end{table}

The attitude is normally specified by its functional and normative roles, and these are usually explained in contrast with those of other attitudes.
E.g. What distinguishes believing from supposing?  These have related roles in guiding action.  Velleman (*ref) suggests that believing differs from supposing in aiming at truth.  We'll return to this idea later.

To specify the content we first need to identify something about its structure.
Mental states are usually thought of as having propositional contents.
But there is a variety of types of content that a mental state can have.
For instance, you can have an attitude towards a map-like structure, an image, an event-type, an object or a relation.

***examples (e.g. use navigation for attitudes towards maps?)

*Explain what propositions are (like numbers).  

*Also explain different types of propositions (Russellian, Fregean \&c)

*illustrate limits of different kinds of content (compare with different kinds of number)


\section{The origin of the attitudes}

Take an attitude like belief or desire.
Suppose someone offers a partial characterisation of the attitude.
For instance, 
suppose they say that belief aims at truth whereas desire aims at satisfaction.
What is this partial characterisation answerable to?
On what grounds should we accept or reject it?

We might treat claims about the attitudes as merely terminological stipulations, so that the only requirement is coherence.
%*rough(!):
This serves only to push back the question further.
What are we attempting to capture in characterising an attitude?  

Another possibility is to think of claims about the attitudes as answerable to ordinary thinking about mental states.
While I doubt we can escape ordinary thinking entirely, I think we should be cautious in appealing to it for two reasons.
One is that we don't actually know very much about how people ordinarily think about mental states.
The other is that ordinary thinking about mental states may not be right, or even consistent.

The approach to the attitudes I prefer is modelling.









\small
\bibliography{$HOME/endnote/phd_biblio_en_record_num_keys}

\end{document}