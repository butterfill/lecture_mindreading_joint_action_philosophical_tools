%!TEX TS-program = xelatex
%!TEX encoding = UTF-8 Unicode

\documentclass[11pt]{extarticle}
% extarticle is like article but can handle 8pt, 9pt, 10pt, 11pt, 12pt, 14pt, 17pt, and 20pt text

\def \ititle {Mindreading \& Joint Action: Philosophical Tools}
\def \isubtitle {Lecture 7: What Is Shared Agency?}
\def \iauthor {Stephen A. Butterfill}
\def \iemail{s.butterfill@warwick.ac.uk}
\date{}

\input{$HOME/Documents/submissions/preamble_steve_handout}

%for strikethrough
\usepackage[normalem]{ulem}

%itemize bullet should be dash
\renewcommand{\labelitemi}{$-$}

\begin{document}

\begin{multicols}{3}

\setlength\footnotesep{1em}

\bibpunct{}{}{,}{s}{}{,}  %use superscript TICS style bib

\bibliographystyle{newapa} %apalike

%\maketitle
%\tableofcontents






\begin{center}
{\Large
Mindreading \& Joint Action: Philosophical Tools}

Lecture 8: Shared Intention \& Motor Representation in Joint Action


ButterfillS@ceu.hu
\end{center}



\section{Question}
What is the relation between a purposive joint action and the goal or goals to which it is directed?

In some cases 
	it is not a shared intention 
	but a special structure of motor representation,
	a `shared motor representation',
	in virtue of which a joint action is related to its goal.


\section{Building blocks}

A \emph{goal} is an outcome to which actions are, or might be, directed.  A \emph{goal-state} is an intention or other state of an agent linking an action to a goal to which it is directed.

%
%\begin{center}
%  \includegraphics[width=0.3\textwidth]{standard_story.png}
%\emph{Figure}: The standard story for individual action.
%\end{center}
%

\textbf{Distributive goal}.  The \emph{distributive goal} of two or more actions is G: (a) each action is individually directed to G; and (b) it is possible that: all actions succeed relative to this outcome.

\textbf{Collective goal}.  The \emph{collective goal} of two or more actions is G:
(a) G is a distributive goal of the outcomes;
(b) the actions are coordinated; and 
(c) coordination of this type would normally  facilitate occurrences of outcomes of G's type


A representation or plan is \emph{agent-neutral} if its content does not specify any particular agent or agents; a planning process is agent-neutral if it involves only agent-neutral representations.


\section{Shared Motor Representation}

We have a \emph{shared motor representation} of an outcome just if 
\begin{enumerate}[label=\emph{\alph*}),itemsep=0pt,topsep=0pt]
\item we each have a motor representation of this outcome; 
\item we are each disposed to inhibit some but not all of the planning or actions resulting from (a);
\item  we each expect that if the outcome occurs, we will all be among the agents of its occurrence; and
\item the truth of (a) and (b) depends on the truth of (c).
\end{enumerate}

%Two or more representations (motor or not) are \emph{reciprocal} just if there is a single outcome which each represents.
%



\section{Evidence that Shared Motor %Rep\textsuperscript{\underline{n}} 
Representation Exists}

In joint action, motor planning can occur for another's actions,\citep{kourtis:2012_predictive} and can inform planning for one's own actions.\citep{vesper:2012_jumping}  %*submitted imagining acting paper

In joint action, it is sometimes necessary to inhibit planning or performing another's action.\citep{sebanz:2006_twin_peaks} 
Whether this is necessary depends on one's beliefs about co-actors' agency.\citep{tsai:2008_action}

In some joint actions, the agents have a single representation of the whole action (not only separate representations of each agent's part).\citep{tsai:2011_groop_effect}


% Vesper says forthcoming EEG paper using piano playing paradigm on agent-neutral identification of error: one brain wave signals whether there is an error, and a different brain wave signals whose error it is (also tells you whether the overall harmonics are affected)





\section{The Interface Problem}
Two  outcomes, A and B, \emph{match} in a particular context just if, in that context, either the occurrence of A would normally constitute or cause, at least partially, the occurrence of B or vice versa. 

A shared motor representation is in \emph{harmony} with a shared intention if they concern matching outcomes.

Some joint actions involve both shared intention and shared motor representation.

How is non-accidental harmony between shared intentions and shared motor representations?

Proposal: `motor imagery could play a crucial role in bridging the gap'\citep{pacherie:2000_content}




\footnotesize 
\bibliography{$HOME/endnote/phd_biblio}

\end{multicols}

\end{document}