%!TEX TS-program = xelatex
%!TEX encoding = UTF-8 Unicode

\documentclass[11pt]{extarticle}
% extarticle is like article but can handle 8pt, 9pt, 10pt, 11pt, 12pt, 14pt, 17pt, and 20pt text

\def \ititle {Joint Action \& the Emergence of Mindreading}
\def \isubtitle {Lecture 2: Minimal Theory of Mind}
\def \iauthor {Stephen A. Butterfill and Ian Apperly}
\def \iemail{s.butterfill@warwick.ac.uk}
\date{}

\input{$HOME/Documents/submissions/preamble_steve_handout}


%itemize bullet should be dash
\renewcommand{\labelitemi}{$-$}

\begin{document}

\begin{multicols}{3}

\setlength\footnotesep{1em}

\bibpunct{}{}{,}{s}{}{,}  %use superscript TICS style bib

\bibliographystyle{newapa} %apalike

%\maketitle
%\tableofcontents






\begin{center}
{\Large
Mindreading \& Joint Action: Philosophical Tools}

Lecture 2: What Are Mental States?


ButterfillS@ceu.hu
\end{center}

'Naturalism in epistemology is merely the attempt to get clear enough about what we mean when we talk about knowledge and perception to be able to tell—in ways a biologist or an experimental psychologist would recognise as scientifically respectable—whether what we are saying is true or not.'\citep%[p.\ x]
{Dretske:2000ky}


\section{Defining belief: normativity}
‘(4) For any p: One ought to believe that p only if p.

‘the holding of this norm is one of the defining features of the notion of belief: it’s what captures the idea that it is constitutive of belief to aim at the truth. The truth is what you ought to believe, whether or not you know how to go about it, and whether or not you know if you have attained it. That, in my view, is what makes it the state that it is.’\citep{boghossian:2003_normativity} %(Boghossian 2003: 37, 38-9)

`belief must be characterized, not just as the attitude having the motivational role, but rather as a truth directed species of that attitude: to believe a proposition is to regard it as true with the aim of thereby accepting a truth.'\citep{Velleman:2000fq} %(247)

‘Aside from our purposes in forming beliefs or in using beliefs as guides to action, there is nothing they should or shouldn’t be.  …  The only fault with fallacious reasoning, the only thing wrong or bad about mistaken judgements, is that, generally speaking, we don’t like them.  We do our best to avoid them.  They do not—most of the time at least—serve our purposes’\citep{Dretske:2000ky} %(Dretske 2000: 247-8)

‘The payments true ideas bring are the sole why of our duty to follow them.  Identical whys exist in the case of wealth and health.  Truth makes no other kind of claim and imposes no other kind of ought than health and wealth do.’\citep{James:1907ae} %(James 1907: 89)

\section{Defining intention: normativity}
`Rational intentions should be agglomerative. If at one and the same time I rationally intend to A and rationally intend to B then it should be both possible and rational for me, at the same time, to intend to A and B.'\citep{bratman_faces_1999} %(Bratman 1999: 220)
 


\footnotesize 
\bibliography{$HOME/endnote/phd_biblio}

\end{multicols}

\end{document}