%!TEX TS-program = xelatex
%!TEX encoding = UTF-8 Unicode

\documentclass[11pt]{extarticle}
% extarticle is like article but can handle 8pt, 9pt, 10pt, 11pt, 12pt, 14pt, 17pt, and 20pt text

\def \ititle {Joint Action \& the Emergence of Mindreading}
\def \isubtitle {Lecture 2: Minimal Theory of Mind}
\def \iauthor {Stephen A. Butterfill and Ian Apperly}
\def \iemail{s.butterfill@warwick.ac.uk}
\date{}

\input{$HOME/Documents/submissions/preamble_steve_handout}


%itemize bullet should be dash
\renewcommand{\labelitemi}{$-$}

\begin{document}

\begin{multicols}{3}

\setlength\footnotesep{1em}

%\bibpunct{}{}{,}{s}{}{,}  %use superscript TICS style bib

\bibliographystyle{newapa} %apalike

%\maketitle
%\tableofcontents






\begin{center}
{\Large
Mindreading \& Joint Action: Philosophical Tools}

Lecture 6: Goal Ascription


ButterfillS@ceu.hu
\end{center}


\section{The Question}
\newcommand{\theQuestion}{How could pure goal ascription work? }
\theQuestion

\newcommand{\dfGoalAscription}{\emph{Goal ascription} is the process of identifying outcomes to which purposive actions are directed as outcomes to which those actions are directed.}

\dfGoalAscription{}

\emph{Pure} goal ascription is goal ascription which occurs independently of any knowledge of mental states.


\section{Obstacle}
Goal ascription involves representing the directedness of an action to an outcome

The relation between actions and outcomes to which the are directed is standardly explained in terms of intention (see Lecture 5).

If directedness could be explained only in terms of intentions or other representations, then pure goal ascription would be impossible---all goal ascription would involve representing representations.

Solution: characterise a relation between actions and outcomes to which the are directed by appeal to teleological functions ...


\section{Teleological Function}

(How to characterise the relation between actions and outcomes to which the are directed without representations.)

Example: 
Atta ants cut leaves in order to fertilize their fungus crops (not to thatch the entrances to their homes) \citep{Schultz:1999ps}

Definition:
`S does B for the sake of G iff: (i) B tends to bring about G; (ii) B occurs because (i.e. is brought about by the fact that) it tends to bring about G.' \citep[p.\ 39]{Wright:1976ls}

Application: 
The Atta ant cuts leaves in order to fertilize iff: (i) cutting leaves tends to bring about fertilizing; (ii) cutting leaves occurs because it tends to bring about fertilizing.


\section{Criteria for a solution}
\theQuestion 
We  seek a relation, $R$, between and action, $a$, and an outcome, $G$, 
such that:
\begin{enumerate}
\item reliably $R(a,G)$ when and only when $a$ is directed to $G$; 
\item $R(a,G)$ is readily detectable; and 
\item $R(a,G)$ is readily detectable independently of any knowledge of mental states.
\end{enumerate}


\section{Can we define $R$ using the Principle of Rationality or Efficiency?}
Principle of Rationality: 
`an action can be explained by a goal state if, and only if, it is seen as the most justifiable action towards that goal state that is available within the constraints of reality' \citep[p.\ 255]{Csibra:1998cx} cf.\ \citep{Csibra:2003jv}.

I.e.: $R(a,G)$ exactly if $a$ is `the most justifiable action towards' $G$ `that is available within the constraints of reality'.

Principle of Efficiency:
`goal attribution requires that agents expend the least possible amount of energy within their motor constraints to achieve a certain end' \citep[p.\ 1061]{Southgate:2008el}.

I.e.: $R(a,G)$ exactly if $a$ is a means of achieving $G$ and any alternative available means would involve expending more energy.

\section{Problems for the Principles}
A. side effects \\
(Many actions have unintended side effects and are rational and efficient ways to produce these side effects.)

B. trade-offs \\
(There is often a balance between how much energy an action would require and how reliably it would achieve a goal.)

C. matching observer and agent \\
(If there are too many discrepancies between
		how well the agent can optimise her actions
	and
		how well the observer can detect optimality,
then these principles will fail to be sufficiently reliable.)


\section{A puzzle}

Motor planning occurs in action observation. \\
Evidence includes findings that observing actions sometimes facilitates performing compatible actions and interferes with performing incompatible actions, as several studies have shown \citep{brass:2000_compatibility, craighero:2002_hand, kilner:2003_interference, costantini:2012_does}. 

Motor planning can facilitate goal ascription. \\
Evidence includes expertise effects \citep{casile:2006_nonvisual},
deficits induced by temporary lesions specifically to the motor cortex \citep{urgesi:2007_representation, moro:2008_neural},
and matches in impairment between performing and identifying actions in patients with
	hemiplegia \citet{serino:2009_lesions_}
	and different apraxias \citep{pazzaglia:2008_sound_}.

\newcommand{\thePuzzle}{How could motor planning in action observation facilitate goal ascription?  }
Puzzle: \thePuzzle 



\section{Planning as goal ascription}
The representation of an outcome leads to a planning-like process which generates predictions about how the action will unfold.
The outcome representation is weakened to the extent that the predictions are not met.  

\theQuestion \\ 
The relation $R(a,G)$ should be defined relative to a planning mechanism.
For planning mechanism $M$, $R{_M}(a,G)$ holds just if were $M$  tasked with producing $G$ it would plan action $a$.


\thePuzzle \\
By enabling the observer to compute whether $R{_M}(a,G)$ where $M$ is the planning mechanism involved in motor control.


\subsection{Solving the problems}

B. trade-offs \\
we get the right principle by deferring to the kinds of planning mechanism responsible for producing the action.

C. matching observer and agent \\
we ensure a match insofar as observer an agent have similar planning mechanisms; this means, of course, that they must have similar expertise

\footnotesize 
\bibliography{$HOME/endnote/phd_biblio}

\end{multicols}

\end{document}