%!TEX TS-program = xelatex
%!TEX encoding = UTF-8 Unicode

\documentclass[11pt]{extarticle}
% extarticle is like article but can handle 8pt, 9pt, 10pt, 11pt, 12pt, 14pt, 17pt, and 20pt text

\def \ititle {Joint Action \& the Emergence of Mindreading}
\def \isubtitle {Lecture 2: Minimal Theory of Mind}
\def \iauthor {Stephen A. Butterfill and Ian Apperly}
\def \iemail{s.butterfill@warwick.ac.uk}
\date{}

\input{$HOME/Documents/submissions/preamble_steve_handout}


%itemize bullet should be dash
\renewcommand{\labelitemi}{$-$}

\begin{document}

\begin{multicols}{3}

\setlength\footnotesep{1em}

%\bibpunct{}{}{,}{s}{}{,}  %use superscript TICS style bib

\bibliographystyle{newapa} %apalike

%\maketitle
%\tableofcontents






\begin{center}
{\Large
Mindreading \& Joint Action: Philosophical Tools}

Lecture 5: Actions, Intentions and Goals


ButterfillS@ceu.hu
\end{center}

\section{Question}
What is the relation between a purposive action and the outcome or outcomes to which it is directed?


\section{What is an intention?}

Davidson's first view: 
`The expression `the intention with which James went to church' has the outward form of a description, but in fact it
...\ % is syncategorematic and 
 cannot be taken to refer to an entity, state, disposition, or event. Its function in context is to generate new descriptions of actions in terms of their reasons; thus `James went to church with the intention of pleasing his mother' yields a new, and fuller, description of the action described in `James went to church'.' (\citeyear[p.\ 690]{davidson:1963_orig}) %
 
Davidson's second view:
`In the case of pure intending, I now suggest that 
the intention simply is an all-out judgement. Forming an intention, deciding, choosing, and deliberating are various modes of arriving at the judgement, but it is possible to come to have such a judgement or attitude without any of these modes applying.' (\citeyear[p.\ 99]{Davidson:1978hy}) %

\subsection{Bratman on Davidson}
`the basic inputs for practical reasoning about what to do---either now or later---will just be the agent's desires and beliefs. 
Such reasoning, when concerned with the future, can issue in future intentions. 
And these intentions are fundamentally different sorts of states from the desires and beliefs on which they are based. 
But there is no significant further role for these intentions to play as inputs into one's further practical thinking. 
Future intentions are, rather, mere spin-offs of practical reasoning concerning the future.' \citep[p.\ 222]{Bratman:1985fk} %page is from reprint

\subsection{Bratman's Objection to Davidson}
A combination of judgements:
%
\begin{quote}
desire: to earn more money

belief: I can earn more money by getting a new job.

judgement: My getting a new job would be desirable.
\end{quote}
and, for each day of your life:
\begin{quote}
desire: to take it easy today

belief: I can take it easy today by not getting a new job today.

judgement: My not getting a new job today would be desirable.
\end{quote}
Making this combination of judgements is not irrational (indeed, both might be correct).
But having the combination of corresponding intentions would be irrational.
Therefore the intentions cannot be the judgements \citep[cf][]{bratman:2000_valuing}.

\section{Norm of Agglomeration}
It is not rational to have several intentions simultaneously unless it is rational to have a single intention agglomerating them all.  


\section{Two Kinds of `Intention'}
`why should rational agents like us have the capacity to have \emph{both} ordinary intentions (subject to demands for consistency and agglomeration) \emph{and} guiding desires (which are not subject to these demands)?
...\ these demands [for consistency and aglomeration] are grounded largely in our needs for coordination.
...\
Our concern with coordination typically obliges us to form intentions, and not merely to allow desires to control our planning and conduct.
This is now always the case, however: intention formation is \emph{but one of several strategies for the resolution of practical conflicts}.'
\citep[][pp.\ 137--8, my italics]{Bratman:1987xw}


\section{Motor representations are like intentions}
Some motor representations (i) represent outcomes, (ii) coordinate actions, and (iii) coordinate actions in ways that would normally facilitate the occurrence of the outcome represented \citep[cf][]{pacherie:2008_action}


\section{Motor representations aren't intentions}
\begin{enumerate}
\item Only representations with a common format can be inferentially integrated.

\item Any two intentions can be inferentially integrated in practical reasoning.

\item My intention that I visit Glasgow on Monday is a propositional attitude.

\item All intentions are propositional attitudes (from 1--3).

\item No motor representations are propositional attitudes.

\item No motor representations are intentions (from 4, 5).
\end{enumerate}





\section{The Interface Problem}
Two  outcomes, A and B, \emph{match} in a particular context just if, in that context, either the occurrence of A would normally constitute or cause, at least partially, the occurrence of B or vice versa. 

Two representations of outcomes are \emph{in harmony} in a particular context if the outcomes they represent match in that context.

Some actions involve both  intention and motor representation.

In some cases, an intention and a motor representation are non-accidentally in harmony.

How is non-accidental harmony ever possible?

A natural way to answer this question would be by appeal to a process of planning or practical reasoning.
But intention and motor representation cannot be inferentially integrated (because they differ in format).  


\section{Demonstrative action concepts}
There are demonstrative concepts of actions.

A demonstrative concept can refer to an action by deferring to a pantomime of an action.

A demonstrative concept can refer to an action by deferring to a \emph{mental} pantomime of an action.

Experiences involved in actually acting, like those involved in mentally pantomiming an action, can ground the possibility of demonstrative reference to action by deference to motor representation.

Some concepts are constituents of intentions and refer to actions by deferring to motor representations.



\footnotesize 
\bibliography{$HOME/endnote/phd_biblio}

\end{multicols}

\end{document}