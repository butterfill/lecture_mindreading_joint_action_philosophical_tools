%!TEX TS-program = xelatex
%!TEX encoding = UTF-8 Unicode

%NB if you change paper size, change it in preamble too (where geometry is loaded)
\documentclass[12pt,a4paper]{extarticle}
% extarticle is like article but can handle 8pt, 9pt, 10pt, 11pt, 12pt, 14pt, 17pt, and 20pt text

\def \ititle {Feedback: Sara Jellinek, `Knowing Another's Thoughts'}
\def \isubtitle {}
\def \iauthor {}
\def \iemail{}
%for anonymous submisison
%\def \iauthor {}
%\def \iemail{}
%\date{}

\input{$HOME/Documents/submissions/preamble_steve_report}

\begin{document}

\setlength\footnotesep{1em}

\bibliographystyle{newapa} %apalike

%these two lines are for anonymous submission --- they remove author and date
%but don't forget to remove defs above as well --- otherwise it will be in the metadata
\author{}
\date{}


\maketitle
%\tableofcontents

% disables chapter, section and subsection numbering
%\setcounter{secnumdepth}{-1} 




This paper is a critical discussion of Donald Davidson's theory of radical interpretation.  As you note, the theory is intended to answer this question: What could someone know that would enable her to understand another's utterances, and how could she come to know it?  One constraint on such a theory is, according to Davidson, that it cannot start from facts about what other's belief or desire.  This is because --- so Davidson --- it is impossible to know what others believe or desire independently of knowing what their utterances mean.  So, according to Davidson, we need a unified theory of meaning and action.  Against this view, you note (p. 3) that we can ascribe beliefs and desires to infants, and to actors in silent movies.  (This happens very quickly; it might have been good to consider whether there is a good reply to these objections.)   

This leads you to reject Davidson's approach in favour of an alternative which depends on `rationality, shared knowledge and circumstances as a basis for the explanation of behaviour' (p. 4).  I would have liked to hear more about the positive theory you offer by way of alternative.  What, on this theory, could we know that would enable us to ascribe to a particular subject in a given context the belief that snow is white (say)?  And what could we know that would enable us to determine that we and another subject have incompatible beliefs concerning the location of a particular object (say)?

Minor points: (1) You write, `It has been proved that infants ... act rationally (Gergely et al 2002).'  Actually these experiments concern whether infants' use the assumption that others act rationally, not whether they act rationally.  (2) In the appendix you discuss figurative language and metaphor, asserting that `neither Davidson's theory ... aimed to explain the process of understanding metaphors' (p. 5).  This is misleading given that Davidson wrote quite a bit about metaphor related issues; see e.g. his paper called `What Metaphors Mean' (1978).

You need to improve presentational aspects.  There are typos in the second and third sentences (`presents --- and later refine' and `what we could know that enable us to interpret') and at least one spelling mistake (`ida', p. 2).  



\small
\bibliography{$HOME/endnote/phd_biblio}

\end{document}