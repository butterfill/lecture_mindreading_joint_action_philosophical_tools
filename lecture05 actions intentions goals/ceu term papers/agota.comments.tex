%!TEX TS-program = xelatex
%!TEX encoding = UTF-8 Unicode

%NB if you change paper size, change it in preamble too (where geometry is loaded)
\documentclass[12pt,a4paper]{extarticle}
% extarticle is like article but can handle 8pt, 9pt, 10pt, 11pt, 12pt, 14pt, 17pt, and 20pt text

\def \ititle {Feedback: Agota Major}
\def \isubtitle {}
\def \iauthor {}
\def \iemail{}
%for anonymous submisison
%\def \iauthor {}
%\def \iemail{}
%\date{}

\input{$HOME/Documents/submissions/preamble_steve_report}

\begin{document}

\setlength\footnotesep{1em}

\bibliographystyle{newapa} %apalike

%these two lines are for anonymous submission --- they remove author and date
%but don't forget to remove defs above as well --- otherwise it will be in the metadata
\author{}
\date{}


\maketitle
%\tableofcontents

% disables chapter, section and subsection numbering
%\setcounter{secnumdepth}{-1} 



This paper is a critical discussion of Kirk Ludwig's paper `Collective Intentional Behavior from the Standpoint of Semantics' (2007).  You start with a series of questions about collective and individual action and what might be required for collective actions to be intentional.  As you write, one key question concerns whether understanding collective intentional behaviour requires an aggregate agent, that is, an agent distinct from any of the individuals who participate in the action where these agents are each proper parts of the plural agent.  I think you start by asking all the right questions.

As you carefully explain, Ludwig's approach is to start with relatively simple sentences involving `we' and work from there.  This approach supports a view about how the term `together' contributes to the meanings of sentences in which it occurs.  Given this view, the truth of ordinary sentences about what we intend do not support the existence of intentions had by aggregate subjects (or, as you put it, `collective intenders'); and the sorts of sentence whose truth would support the existence of such intentions are, as you say, odd (p. 3).  Here you make a move that puzzles me.  You write, "Ludwig thus concludes that there are no collective intenders ... nor is there collective intending".  The puzzle is how we get from a discovery about semantics to a conclusion about what exists.  I wish you had said more about this move.

The first criticism you offer of Ludwig's view hinges on the assertion that "in order to qualify an action as a joint action ... some kind of coordination ... is also needed" (p. 3).  I think this is an interesting objection to pursue.  But as you present it, you merely assert that coordination is needed for joint action.  Of course Ludwig has to deny this.  So there is simply a stand-off between you: you say coordination is needed, Ludwig says it isn't.  How can we make progress on this issue?

Later, on p. 5, I interpret you as asserting that, in intentional joint action, "all members of the group intend that every member of the group is an agent of the action".  It would be good to know why this is true, particularly as you later discuss an example where it seems false or at least not obviously true: voting for president.  Couldn't I take part in the activity of electing a president even without having any idea who else is involved?  (You move freely from small- to large-scale cases of joint action; this is always going to be tricky.)

Overall I am impressed with your analysis.  You have done a good job of getting to grips with, and carefully criticising, some difficult material.  How could it be better?  A key limit of your paper is that it focussed exclusively on a single paper (although I know that you're drawing on more).  It might be good to explain consequences of your conclusions for other views, e.g. for proponents of a view along the lines of Searle (1983).  It might also be tempting to think, given what you have written, that Ludwig's analysis \emph{merely} concerns semantics.  Does it inform any novel predictions?  Could any of this matter for understanding which mechanisms make joint action possible?

Minor point: on p. 2 you refer to a `collaborative reading' whereas the correct term is `collective reading'.


\small
\bibliography{$HOME/endnote/phd_biblio}

\end{document}