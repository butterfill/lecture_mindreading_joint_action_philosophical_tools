%!TEX TS-program = xelatex
%!TEX encoding = UTF-8 Unicode

%NB if you change paper size, change it in preamble too (where geometry is loaded)
\documentclass[12pt,a4paper]{extarticle}
% extarticle is like article but can handle 8pt, 9pt, 10pt, 11pt, 12pt, 14pt, 17pt, and 20pt text

\def \ititle {Feedback: Denis Tatone, `Joint Action and Teleological Action Understanding’}
\def \isubtitle {}
\def \iauthor {}
\def \iemail{}
%for anonymous submisison
%\def \iauthor {}
%\def \iemail{}
%\date{}

\input{$HOME/Documents/submissions/preamble_steve_report}

\begin{document}

\setlength\footnotesep{1em}

\bibliographystyle{newapa} %apalike

%these two lines are for anonymous submission --- they remove author and date
%but don't forget to remove defs above as well --- otherwise it will be in the metadata
\author{}
\date{}


\maketitle
%\tableofcontents

% disables chapter, section and subsection numbering
%\setcounter{secnumdepth}{-1} 



How much conceptual sophistication must  agents have in order to engage in joint action?
This rich, ambitious and thoughtful paper argues that the agents would need only to be able to apply teleological reasoning to plural activities.%
\footnote{
Here and throughout \emph{teleological reasoning} refers to the notion characterised by Gergely and Csibra.
For agents to be engaged in a \emph{plural activity} it is sufficient that each agent’s actions are individually organised around a single outcome which occurs as a common effect of all the agents’ actions.
}


Many researchers on joint action have assumed some kind of common knowledge requirement.
Simply dropping this requirement appears to create a problem, for its absence seems to mean agents involved in a joint action might have ‘no evidence to justify the decision to engage in a potentially costly activity’ (p. 3).  
This is a promising line of objection to accounts like Pacherie’s and I would like to see it developed further.
As things stand, I think Pacherie might reply by saying that the question of justification is orthogonal to that of whether someone is engaging in joint action.
By comparison, someone might engage in jumping a chasm without any evidence to justify to doing so.
The lack of justification does not seem to bear directly on the question of whether they are jumping a chasm.
Similarly, it is not yet obvious why someone’s assuming without justification that another will cooperate means that they are not engaged in joint action.
But perhaps Denis would agree.
It remains a good question---and one unanswered by the account criticised, and indeed unanswered by the other accounts discussed too---how someone might be justified in deciding to engage in joint action.


The paper moves from this question about justification to another, related question.
Under what conditions can individuals be regarded as agents of an event? 
The proposed answer involves teleological reasoning.
I find it difficult to be confident in attempting to formulate the principle because it is explained only in terms of an example.
The example is excellent, but I would like to see an abstract statement of the view too.
One possibility that seems mostly consistent with the paper is this: for some individuals to be agents of an event is for the event to consist in each of them acting in such a way that they all act in a way that is a most rational way for them collectively to achieve some outcome; and where this occurs, the outcome is a shared goal of their actions.

In developing this idea, the author hits on a useful idea. 
As mentioned earlier in the paper (pp. 3--4), the author focuses on cases in which the involvement of agents in a joint action is not manifest in advance of their acting but rather becomes apparent ‘on the fly’. 
The useful idea is that Matt can make manifest to Bob the fact that I Matt engaging in joint action with Bob simply by assuming that they will act jointly and doing what his part requires. 
This is not explicitly connected to the issue about justification mentioned earlier, but seems to bear on it in important ways.
I shall thing more on this, and I think it would be a good point to develop.

I was also impressed by the critique of Hamann’s and Tomasello’s claims that a subject’s failure to understand normative aspects of joint action (as manifested by absences of mutual support and sharing) implies that she is not engaged in joint action.
Again, though, I think it would be good to go more slowly here.
There are some subtle issues.
Further, as a general rule it is good to explain how the target of an objection might respond.

Overall my main criticism of this excellent paper is also a complement: despite clear writing, good use of examples and lots of scholarship, too much happens too quickly.
 

PS: Does Carpenter really claim that engaging in joint action fosters mindreading (as you say on p.\ 1)?



\section{Typos}
p.\ 2: ‘process’ for ‘proceed’

p.\ 5: ‘defined [as] executive’

p.\ 6: ‘relatively’ for ‘relative’

p.\ 6: ‘bind’ for ‘bound’






\small
\bibliography{$HOME/endnote/phd_biblio}

\end{document}