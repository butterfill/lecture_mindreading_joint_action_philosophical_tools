%!TEX TS-program = xelatex
%!TEX encoding = UTF-8 Unicode

%NB if you change paper size, change it in preamble too (where geometry is loaded)
\documentclass[12pt,a4paper]{extarticle}
% extarticle is like article but can handle 8pt, 9pt, 10pt, 11pt, 12pt, 14pt, 17pt, and 20pt text

\def \ititle {Feedback: Pavel Voinov, `Where Do the Hypothetical Limits of Non-human Animal Abilities for Joint Action Lie?’}
\def \isubtitle {}
\def \iauthor {}
\def \iemail{}
%for anonymous submisison
%\def \iauthor {}
%\def \iemail{}
%\date{}

\input{$HOME/Documents/submissions/preamble_steve_report}

\begin{document}

\setlength\footnotesep{1em}

\bibliographystyle{newapa} %apalike

%these two lines are for anonymous submission --- they remove author and date
%but don't forget to remove defs above as well --- otherwise it will be in the metadata
\author{}
\date{}


\maketitle
%\tableofcontents

% disables chapter, section and subsection numbering
%\setcounter{secnumdepth}{-1} 

This illuminating and insightful paper sets out to refute 
\citet[p.\ 2027]{Knoblich:2008hy}’s assertion that ‘the full machinery for joint intentionality ... needs to be in place before the know-how about tools can be passed on between individuals or generations.’
I was puzzled about why they made this claim when I first read that paper, so I was immediately gripped by Pavel’s paper.
A helpful review of the literature reveals much evidence that know-how about tools can be passed between both individuals and generations in many non-human species.
Further, there is also evidence that some non-humans can learn by imitation, even  if imitative learning does not occur in their natural environments.
Finally, several non-human species follow gaze cues, are sensitive to what others can and cannot see, succeed on competitive level-1 perspective taking tasks and grab other’s attention by using attention getters including gaze alternation.
At the very least, this makes it hard to be sure that they are incapable to joint attention (p.\ 8).

One possible objection is that the author does not examine the assumption that the various non-human species considered lack ‘the full machinery for joint intentionality’.
While this might seem hardly necessary at the outset, the paper’s review of non-human competence invites readers to question that assumption.
And of course if the assumption is wrong, Knoblich and Sebanz’ claim about transmission of tool-use know-how being dependent on joint intentionality will not have been refuted.

Pavel appears to dismiss the possibility that non-humans possess ‘the full machinery for joint intentionality’ on the grounds that they are not ‘as competent in their cooperative abilities as humans are’ (p.\ 4). 
However, he points out, correctly, I think, that non-humans might be capable of imitation although they do not exhibit imitation in their natural environments (p.\ 5).
It seems to me that a parallel point about shared intentionality can be made.
Indeed, Pavel appears to consider this possibility towards the end (p.\ 8) when considering a possible ‘lack of motivation’.
(Incidentally, I don’t think the point about lack of motivation is the same as the point about lacking an ability to see others as agents like oneself.
And isn’t that ability indicated by the use of gaze alternation and attention getters?)

What could be better?
The writing is good but there is still plenty of room for improvement.
I could not interpret the final sentence of the last compete paragraph on page 6 (‘Though, authors ...’).
I was also concerned that there are some undefended assertions in the first paragraph:
%
\begin{enumerate}
\item ‘abilities to engage in joint action ... are not a dedicated function or process’
\item ‘criteria for inclusion [in the class of joint activities] are a matter for convention’
\end{enumerate}
%
The first claim is trivially correct taken literally if (as I suppose), abilities are not functions or processes.
The author intends to deny that the abilities involve dedicated processes; but I don’t think this denial is obviously correct.
The second claim also appears to need defence.
After all, to claim that which events  are actions or explosions is ‘a matter for convention’ would be to push quite a radical line.

More substantively, a weakness of the paper is that it sets out to refute an assertion for which no argument is offered.
To show that refutation is worthwhile, it would be good to offer at least a \emph{prima facie} persuasive defence of the claim.
After all, the fact that several eminent researchers assert something doesn’t automatically make that assertion worth scrutinising.
But this is a minor criticism because the central points of the paper are of great interest, at least to this reader.







\small
\bibliography{$HOME/endnote/phd_biblio}

\end{document}