%!TEX TS-program = xelatex
%!TEX encoding = UTF-8 Unicode

%NB if you change paper size, change it in preamble too (where geometry is loaded)
\documentclass[12pt,a4paper]{extarticle}
% extarticle is like article but can handle 8pt, 9pt, 10pt, 11pt, 12pt, 14pt, 17pt, and 20pt text

\def \ititle {Feedback: Caglan Dilek, ‘Bratman’s Shared Cooperative Activity and the Nature of Joint Action’}
\def \isubtitle {}
\def \iauthor {}
\def \iemail{}
%for anonymous submisison
%\def \iauthor {}
%\def \iemail{}
%\date{}

\input{$HOME/Documents/submissions/preamble_steve_report}

\begin{document}

\setlength\footnotesep{1em}

\bibliographystyle{newapa} %apalike

%these two lines are for anonymous submission --- they remove author and date
%but don't forget to remove defs above as well --- otherwise it will be in the metadata
\author{}
\date{}


\maketitle
%\tableofcontents

% disables chapter, section and subsection numbering
%\setcounter{secnumdepth}{-1} 

This paper discusses Bratman’s account of shared cooperative activity in relation to four questions about joint action (p. 1).  
Several criticisms are made, including: first, that coercion is not  incompatible with shared cooperative activity (pp.\ 8, 1--11); second, that the individualistic approach is inadequate (p. 10); and third that ‘should be able to explain such performances which involve quick and difficult sub-actions’ (p.\ 10).

What could be improved? 
The criticisms of Bratman’s account need more extended development.
To take just one illustration, the notion of a sub-intention appears to be pivotal to your discussion (‘it is more enlightening to talk of sub-intentions rather than sub-plans’, p.\ 11) but you do not explain what a sub-intention is.
This notion is used in the footballers example (p.\ 10--11) but I wasn’t able to work out what you have in mind.
A better approach would be to focus on a single line of criticism and to develop it in detail.

It is important in offering objections to consider how the target view might be refined to avoid the objection.
This will also enable you to sharpen your objections.
For instance, your point about coercion (p.\ 8) does not show that coercion is not ‘relevant’ to shared cooperative activity; it just shows that shared cooperative activity does not invariably preclude coercion of any kind.

The writing is good but there is still plenty of room for improvement.
For instance, there are too many thoughts in this sentence, which also has syntactic defects:
\begin{quote}
‘We do involve in joint actions and some researchers argue that lots of cognitive abilities evolved because we act cooperatively, maybe even self-­‐consciousness is the result of our abilities of joint action, if we take mindreading and goal ascription as related abilities’  (p. 1).
\end{quote}
And this sentence has syntactic defects that make it hard to understand:
\begin{quote}
‘I think there is more why they are not collective actions’ (p. 5).
\end{quote}

I was very interested in two points.
First, the paper concludes with the interesting suggestion that minimal accounts of joint action might fail to explain ‘distinctive features of joint action’ (p.\ 12). 
It would be good to say more about this---which features are left unexplained, and why exactly?
Second, there is (too brief) presentation of an objection to individualistic approaches (like Bratman’s): 
\begin{quote}
‘once he starts from a individualistic intentions and planning, it is hard to explain through them the later occurring shared intentions, which should not be different from the starting individual intentions. It is hard that all the intentions and planning continue to exist till the end without losing their explanatory power on joint action.’ (p.\ 10)
\end{quote}
I was unable to understand this suggestion, but it sounds interesting and is perhaps worth developing.






\small
\bibliography{$HOME/endnote/phd_biblio}

\end{document}