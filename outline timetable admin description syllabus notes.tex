%!TEX TS-program = xelatex
%!TEX encoding = UTF-8 Unicode

\def \papersize {a4paper}
\documentclass[12pt,\papersize]{extarticle}
% extarticle is like article but can handle 8pt, 9pt, 10pt, 11pt, 12pt, 14pt, 17pt, and 20pt text

%Mindreading and Joint Action: Philosophical Tools 
%(Fall 2012-3, PhD Elective)
%lecturer: S. Butterfill

\def \ititle {Mindreading and Joint Action: Philosophical Tools}
\def \isubtitle {}
\def \iauthor {CEU, Fall 2012-3, PhD Elective}
\def \iemail{s.butterfill@warwick.ac.uk}
%for anonymous submisison
%\def \iauthor {}
%\def \iemail{}
%\date{}

\input{$HOME/Documents/submissions/preamble_steve_paper3}

%avoid overhang
\tolerance=5000


\begin{document}

\setlength\footnotesep{1em}

\bibliographystyle{newapa} %apalike

%these two lines are for anonymous submission --- they remove author and date
%but don't forget to remove defs above as well --- otherwise it will be in the metadata
%\author{}
%\date{}


\maketitle
%\tableofcontents
%
%\begin{abstract}
%\noindent
%***
%\ 
%
%\end{abstract}




\section{Short course description}
My aim is to introduce research on philosophical topics which is likely to be useful to cognitive scientists with interests in joint action or mindreading. 
In selecting topics I apply three criteria.
The topics: (a) are usually neglected (or misunderstood) by cognitive scientists;
(b) might usefully inform the design and interpretation of experiments;
and
(c) ***


***How is mindreading involved in joint action?  And in what ways (if any) could mindreading, or its development or evolution, depend on abilities to engage in joint action?  

***



\section{Provisional schedule}

Most classes will take the form of a lecture with questions and discussion time. 
There are also four afternoon sessions in which we'll discuss key papers.
See table \vref{table:schedule} for a provisional schedule. 
The schedule may change depending on group discussion and research interests.

{
	%increase space between rows
	\renewcommand{\arraystretch}{1.5}
\begin{table}[htbp]
\begin{center}
\footnotesize	%shrink for better spacing
\begin{tabular*}{1\textwidth}{ l l m{0.80\textwidth} } 

\toprule

1. & sept 12  
	&  \textit{Introduction: Three Puzzles about Mindreading and Joint Action}
\\  2. & sept 19  
	& \textit{What Are Mental States?}
		\newline Reading: \citet[§§1.1--1.3, 3.1--3.4, 4.1]{Jeffrey:1983oe}, \citet[§1]{fitch:2009_singular}
\\ &   sept 26 & [no class]
\\ 3. & oct 3 
		& \textit{Tracking, Measuring and Representing Beliefs}
			\newline  Reading: \citet{matthews:1994_measure}, \citet{kovacs_social_2010}
\\  4. & oct 10 
	&  \textit{What is Core Knowledge (or Modularity)?}
		%this is where I can do the paradox about the ages]
		\newline Reading: \citet{Fodor:1983dg,Fodor:2000cj,Baillargeon:gx,Wellman:2001lz}
		\newline Discussion: {\citet{Sugden:2000mw}}
\\ & oct 14 & [no class]
\\ 5. & oct 24 
	& 	\textit{Radical Interpretation}
		\newline Reading: \citet{Davidson:1985qg,Davidson:1973jx,Davidson:1980xp,Davidson:1990du}
		\newline Discussion: \citet{matthews:1994_measure}
\\ 6. & oct 31 
	& \textit{Actions, Intentions and Goals}
		\newline Reading: \citet{Davidson:1971fz,Davidson:1978hy,Bratman:1985fk,bratman:2000_valuing} 
			%*Bratman What is Intention? (hard to source)
		\newline Discussion: \citet{Davidson:1973jx}
\\ & nov 7 & [no class]
\\ 7. & nov 14 
	& \textit{Goal Ascription: the Teleological Stance and Motor Awareness}
		\newline Reading: \citet{Millikan:1989cd,Millikan:1993_behaviour,Millikan:1993_green,Gergely:1995sq,Csibra:2003kp,Fogassi:2005nf}
		\newline Discussion: \citet{Bratman:1984jr}
\\ 8. & nov 21 
	&  \textit{What Is Joint Action?}
		\newline Reading: \citet{Bratman:1992mi,Bratman:1993je,ludwig_collective_2007,Searle:1990em}
		%\newline Discussion: *
\\ & nov 28 & [no class]
\\ 9. & dec 5 
	& \textit{Shared Intention and Motor Representation in Joint Action}
	%[was planning to do Interacting Mindreaders but need to show some results (and will have done Interacting Mindreaders at the CEU colloq.]
		\newline Reading: \citet{Knoblich:2008hy,kourtis:2010_favoritism}
\\
%
\bottomrule
%
\end{tabular*}
\caption{Provisional schedule}
\label{table:schedule}
\end{center}	%careful -- position of this affects distance between table and caption(!)
\end{table}
}



\section{Method of evaluation}

Students may be asked to prepare one or more short (5--10 minute) presentations to introduce a discussion sessions. 
Presentations will not be graded nor contribute to any overall mark. 

Students may submit a short midterm paper of no more than 3000 words (fewer is better).
Midterm papers will not contribute to any overall mark.   
Midterm papers will receive feedback some feedback (a paragraph or so) but will not receive grades.

Students should submit a short term paper.  
The topic of each paper should ideally be agreed in advance; alternatively term papers may answer a question chosen from the list of questions below. 
Term papers may not substantially overlap with midterm papers where any individual is an author of both.


\section{Sample essay questions}
In addition to those below, also consider questions from the titles of lectures in table \vref{table:schedule}.
%
\begin{enumerate}
\item Are there limits on the behaviour that can be modelled using simple forms of decision theory \citep[such as the version presented in][]{Jeffrey:1983oe}?  You may choose to answer with respect to one of \citet{Sugden:2000mw} or \citet{bratman:2000_valuing}.
\item How, if at all, can we distinguish different kinds of mindreading?  If you provide a distinction, discuss an application of it.
\item Which events are actions?
\item What is the relation between a goal and an action when the action is directed to the goal?
\item What could someone represent that would enable her to track others' desires?
\item What could count as evidence that a mindreader was ascribing intentions to other individuals?  You might relate your answer one or more of the following: \citet{fogassi_mirror_2007,Dasser:1989qw,astington:2001_paradox,malle:2001_distinction}.
\item `The concept of a joint action as such is just that of an event of which there are multiple agents' \citep[p.\ 366]{ludwig_collective_2007}.  First explain and then evaluate this claim.
\item Does joint action necessarily involve mindreading?
\end{enumerate}


\section{Deadlines for submitting papers}

\subsection{Midterm papers}
Midterm papers should be submitted by 9 am on Monday 5th November.  Mindterm papers should be emailed directly to me (Butterfill).  Late midterm papers will not be read without prior agreement.  Midterm papers will be returned with feedback by Wednesday November 14, 2012 (unless there are very many).  

\subsection{Term papers}
The deadline for submitting term papers is 9 am on January 7, 2012. The grades will be returned by the instructor by January 21, 2013 together with some feedback (a paragraph or so).



\section{Reading and Sources}

(See the provisional schedule in table \vref{table:schedule}.)

{
	\def\section*#1{}	%remove section heading
	\bibliography{$HOME/endnote/phd_biblio}
}









\end{document}