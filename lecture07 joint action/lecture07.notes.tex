%!TEX TS-program = xelatex
%!TEX encoding = UTF-8 Unicode

\def \papersize {a4paper}
\documentclass[12pt,\papersize]{extarticle}
% extarticle is like article but can handle 8pt, 9pt, 10pt, 11pt, 12pt, 14pt, 17pt, and 20pt text

\def \ititle {Mindreading and Joint Action: Philosophical Tools}
\def \isubtitle {Notes for Lecture 7 \\ What Is Shared Agency?}
\def \iauthor {Stephen A. Butterfill}
\def \iemail{ButterfillS@ceu.hu}
%for anonymous submisison
%\def \iauthor {}
%\def \iemail{}
%\date{}

\input{$HOME/Documents/submissions/preamble_steve_paper3}

%avoid overhang
\tolerance=5000


\begin{document}

\setlength\footnotesep{1em}

\bibliographystyle{newapa} %apalike

%these two lines are for anonymous submission --- they remove author and date
%but don't forget to remove defs above as well --- otherwise it will be in the metadata
%\author{}
%\date{}


\maketitle
%\tableofcontents
%
%\begin{abstract}
%\noindent
%***
%\ 
%
%\noindent
%\textbf{Keywords:}
%Mindreading, Joint Action, Action, Belief, Intention, Representation, Mental State
%\end{abstract}
%



\section{Introduction}
What is shared agency?
Although shared agency is a feature of many everyday events,
 it isn't easy to say what it is.
One way to get an intuitive, pre-theoretical fix on phenomena of shared agency is to look at ways in which
	an exercise of shared agency 
	can contrast with
	superficially similar exercises of individual agency in parallel.

When members of a flash mob in the Central Cafe respond to a pre-arranged cue by noisily opening their newspapers, they exercise shared agency. 
But when someone not part of the mob just happens to noisily open her newspaper in response to the same cue, her action does not involve shared agency.%
\footnote{
See \citet{Searle:1990em}; in his example park visitors simultaneously run to a shelter, in once case as part of dancing together and in another case because of a storm. 
Compare \citet{Pears:1971fk} who uses contrast cases to argue that whether something is an ordinary, individual action depends on its antecedents. 
}
To give another example, 
 two members of the mob exercise shared agency when they 
   later walk to the metro station together as friends. 
But two people who merely happened to be walking to the metro station side by side would not be exercising shared agency \citep{gilbert_walking_1990}. 
These and other contrast cases invite the question, 
\textbf{
How do activities involving shared agency differ from activities involving parallel agency only? }


\subsection{Why it matters [\emph{*skip---this breaks the flow}]}
Before taking this question head-on,
it's helpful to constrain what would count as a good answer by
stepping back and asking,
	Why is it useful 
	to  distinguish systematically between shared and individual agency?
	
Shared agency raises a tangle of scientific and philosophical questions.  Psychologically we want to know which mechanisms make it possible \citep{Sebanz:2006yq,vesper_minimal_2010}.  
Developmentally we want to know when shared agency emerges, what it presupposes and whether it might somehow facilitate socio-cognitive, pragmatic or symbolic development \citep{Moll:2007gu,Hughes:2004zj,Brownell:2006gu}.  
Phenomenologically we want to characterise what (if anything) is special about experiences of action and agency when shared agency is involved \citep{Pacherie:2010fk}.  
Metaphysically we want to know what kinds of entities and structures are implied by the existence of shared agency \citep{Gilbert:1992rs,Searle:1994lb}.  
And normatively we want to know what kinds of commitments (if any) are entailed by shared agency and how these commitments arise \citep{Roth:2004ki}.
%, plus a formal account of how practical reasoning for joint action differs (if at all) from individual practical reasoning \citep{Sugden:2000mw,Gold:2007zd}
A philosophical account of shared agency 
that made it possible to systematically distinguish shared from individual agency  
may support investigation of these questions (as \citealp{Bratman:2009lv} suggests).  
%Or, if there is more than one form of shared agency, the account should provide a principled way of distinguishing among forms of individual and shared agency.   


\subsection{Back to the Question}
Let us return to the contrast cases.

How do cases involving shared agency differ from cases involving parallel agency only? 

The first example shows that the difference between shared agency and parallel individual agency can’t just be that the resulting actions have a common effect because merely parallel actions can have common effects too. 
And the second example shows that the difference can’t just be a matter of coordination, because people who are merely happen to be walking side by side each other also need to coordinate their actions in order to avoid colliding.  
Note also that in both cases each individual's walking is intentional, so our intentionally walking together cannot be  only a matter of our each intentionally walking.%
\footnote{
This use of contrast cases resembles \citet{Pears:1971fk}: he uses contrast cases to argue that whether something is an ordinary, individual action depends on its antecedents. 
} 

Perhaps, then, a notion of shared intention is needed to distinguish the two cases.  
Perhaps it is our acting on a shared intention that we walk together which distinguishes us from two strangers who happen to be walking side by side.

This is view is common among philosophers psychologists:
%
\begin{quote}
	`I take a collective action to involve a collective intention'  \citep[p.\ 5]{Gilbert:2006wr}. 
\end{quote}	
%
\begin{quote}
`the key property of joint action lies in its internal component [...] in the participants' having a ``collective'' or ``shared'' intention' \citep[pp.\ 444-5]{alonso_shared_2009}.
\end{quote}
%
\begin{quote} 
`The sine qua non of collaborative action is a joint [shared] goal and a joint commitment’ 
(Tomasello 2008, p. 181).
\end{quote} 
%
\begin{quote}
`Shared intentionality is the foundation upon which joint action is built' \citep[p.\ 381]{Carpenter:2009wq}.
\end{quote}
%
\begin{quote}
`it is precisely the meshing and sharing of psychological states \ldots \ that holds the key to understanding how humans have achieved their sophisticated and numerous forms of joint activity'
\citep[p.\ 369]{Call:2009fk}.
\end{quote}
%
But what could shared intention be?



\section{Shared Intention}
What is shared intention?
Here is much disagreement.
Some hold that it differs from
ordinary intention with respect to the attitude involved (\citealp{Searle:1990em}). 
Others have explored the notion that it differs from ordinary intention with respect to its subject, which is plural \citep{Gilbert:1992rs,helm_plural_2008}, 
or that it differs from ordinary intention in the way it arises, namely through team reasoning \citep{Gold:2007zd}, 
or that it involves distinctive obligations or commitments to others (\citealp{Gilbert:1992rs}; \citealp{Roth:2004ki}).
Opposing all such views, \citet{Bratman:1992mi,Bratman:2009lv} argues that shared intention can be realised by multiple ordinary individual intentions and other attitudes whose contents interlock in a distinctive way. 

To avoid having to take sides on what shared intention is, 
let us abstract the details
and focus on features that everyone should agree on.
Minimally, a shared intention stands to an activity involving shared agency  in roughly the way that an ordinary, individual intention stands to an ordinary, individual action.
%
\begin{quote}
Shared intention stands to action involving shared agency as an ordinary intention stands to an ordinary intentional action.
\end{quote}
%
Let us develop this parallel.
Concerning ordinary individual agency we can ask,
\begin{quote}
What is the relation between a purposive action and the goal or goals to which it is directed?
\end{quote}
As we saw last week, the notion of intention provides one way to answer this question.
For an ordinary, individual intention represents an outcome, coordinates an agent's actions, and coordinates the agent's actions in such a way that, normally, this coordination would facilitate the occurrence of the represented outcome.

Concerning shared agency there is a parallel question:
%
\begin{quote}
What is the relation between a purposive joint action and the outcome or outcomes to which it is directed?
\end{quote}
%
Here and onwards I'm going to use the term \emph{joint action} to mean an exercises of shared agency.
(This is just a stipulation; others use the term differently \citep[e.g.][]{ludwig_collective_2007}.)

Just as we answered the original question by appeal to intention,
so we can answer this question about joint action by appeal to shared intention.
A shared intention involves there being a single outcome which each agent represents, where these representations coordinate the several agents' actions and coordinate them in such a way that, normally, the coordination would facilitate the occurrence of the represented outcome. 


There is another feature of shared intention that should be uncontroversial on any account of it.  
Shared intentions or their components feature in practical reasoning alongside ordinary, individual intentions,
 and there are normative requirements which apply to combinations of individual and shared intentions.
To illustrate, tonight there is a party and a ceremony.
It is impossible for anyone to attend both, and this is common knowledge among us.
We have a shared intention that we attend the ceremony together.
While having this shared intention, I also intend to go to the party. 
Give our common knowledge,
this combination of shared and individual intentions is irrational. 
Its irrationality is related to that which would be involved in my individually intending to attend the ceremony while also intending to go to the party.
In short, shared intentions or their components are inferentially and normatively integrated with ordinary, individual intentions.

It's important to see that we have been \textbf{moving in a circle}.
The question was, What is shared agency?  
We're following most philosophers in supposing that we can answer this question by appeal to shared intention.
That is, an activity involves shared agency when it involves actions that stand in an appropriate relation to a shared intention.  
Then we asked what  shared intention is.  
And the answer is, it’s something that stands in an appropriate relation to activities involving shared agency.  
I don’t think all circles are bad, and this circle has provided us with some constraints on shared intentions could be.
But it's important to see that, by moving in a circle, we have not said positively what shared intention is.
For all we have said so far, it might be a mistake to assume that there is anything which stands to joint action as ordinary, individual intention stands to ordinary, individual action.

We need an account of shared intention.


\section{Bratman on Shared Intention}
Although there are many accounts of shared intention,
the best developed account,
to which no one has yet pushed a counterexample or decisive objection,
is Michael Bratman's.
***

\begin{quote}
\label{quote:bratman_account}
`1. (a) I intend that we J and (b) you intend that we J
 
`2. I intend that we J in accordance with and because of la, lb, and meshing subplans of la and lb; you intend that we J in accordance with and because of la, lb, and meshing subplans of la and lb
 
`3. 1 and 2 are common knowledge between us' \citep[][p.\ View 4]{Bratman:1993je}
\end{quote}
%



\section{A challenge}
\label{sec:objection_to_bratman}
Is Bratman's account of shared intention and shared agency adequate to our needs?

One problem is that shared intentions are cognitively and conceptually demanding.

Quote Guenther and Natalie:
%

\begin{quote}
`the contribution of lower-level processes to social interaction has hardly been considered. This has led philosophers to postulate complex intentional structures that often seem to be beyond human cognitive ability in real-time social interactions.'
\citep[p.\ 2022]{Knoblich:2008hy}
\end{quote}
%
This claim needs careful defence.

***Defend the claim that shared intention is conceptually and cognitively demanding.

\begin{quote}
`shared intentional agency consists, at bottom, in interconnected planning agency of the participants' \citep[p.\ 11]{Bratman:2011fk}.
\end{quote}
%
So Guenther and Natalie are right insofar as the account is conceptually and cognitively demanding.

But so what if, on Bratman’s account, shared intention requires conceptual and cognitive sophistication?
Maybe this is just how it is with joint action.
So I don’t think there’s an objection to the account here yet.
There’s just a question.
The question is whether there are forms of shared agency which are not demanding in the way that Bratman’s account suggests.

There's a further reason why we might want a less demanding kind of shared agency ... 

Several researchers have conjectured that shared agency in some sense grounds sophisticated forms of cognition.
In this vein, Sebanz and Knoblich conjecture that:
%
\begin{quote}
`[F]unctions traditionally considered hallmarks of individual cognition originated through the need to interact with others ...\
perception, action, and cognition are grounded in social interaction.' \citep[p.\ 103]{Knoblich:2006bn}
\end{quote}
%
%If this is right, there must exist forms of social interacting that (i) can `ground' sophisticated forms of perception or cognition and (ii)  don't already presuppose those very forms of cognition.
Relatedly, Moll and Tomasello offer a `Vygotskian Intelligence Hypothesis':
\begin{quote}
`the unique aspects of human cognition ... were driven by, or even constituted by, social co-operation.' \citep[p.\ 1]{Moll:2007gu}
\end{quote}
%
And Sinigaglia and Sparaci write:
\begin{quote}
`human cognitive abilities ... [are] built upon social interaction' 
(*ref Sinigaglia and Sparaci 2008)
\end{quote}
%
Although these researchers write about `cultural interactions' and `social interactions', I think it's clear from the contexts that their target is shared agency.

These bold ideas are the inspiration for a more limited conjecture.
%
\begin{quote}
\emph{Conjecture} 
The prior existence of capacities for shared agency partially explains how sophisticated forms of mindreading emerge in evolution or development (or both).
\end{quote}
%
This conjecture faces several objections.
First, don't don't sophisticated forms of mindreading emerge before shared agency?
We've already dealt with this objection by distinguishing minimal from full-blown mindreading in earlier lectures.
The discussion from those lectures does not show that the objection is actually mistaken; but it does show that whether the objection holds good or not depends on further empirical research.

The second objection is relevant now.  It is this:
\begin{enumerate}
\item 
	All shared agency involves shared intention.
\item 
	Shared intention requires sophisticated mindreading.
\end{enumerate}
%
Therefore:
%
\begin{enumerate}[resume]
\item 
	Shared agency could play no significant role in explaining how sophisticated forms of mindreading emerge.
\end{enumerate}
%
In response to this argument some philosophers have attempted to reject the second premise \citep[e.g.][]{Tollefsen:2005vh, Pacherie:2012fk}.   
This would mean rejecting Bratman's account of shared intention, and formulating and defending an account.
This would be very difficult because Bratman's account has proved quite resistant to objections.

Instead of taking this route,
I want to consider rejecting the first premise, (1) above.

Is it reasonable to reject the first premise?
You might think not, 
for this reason:
As philosophers, the topic is not just doing things together but intentionally doing them together.%
\footnote{
I.e. here we argue about \emph{intentional joint action}. ***
}
But even if we accept this, I think we shouldn’t assume that all shared agency involves shared intention.
After all, the parallel with ordinary, individual agency suggests otherwise.
[*More on this in Budapest version of \emph{Shared Agency and Motor Representation} paper.]


\section{Missing bits}
See the \emph{What Is Joint Action? A Modestly Deflationary Account (Shared and Collective Goals)} paper.


\section{Conclusion}




\small
\bibliography{$HOME/endnote/phd_biblio}

\end{document}