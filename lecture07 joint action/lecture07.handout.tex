%!TEX TS-program = xelatex
%!TEX encoding = UTF-8 Unicode

\documentclass[11pt]{extarticle}
% extarticle is like article but can handle 8pt, 9pt, 10pt, 11pt, 12pt, 14pt, 17pt, and 20pt text

\def \ititle {Mindreading \& Joint Action: Philosophical Tools}
\def \isubtitle {Lecture 7: What Is Shared Agency?}
\def \iauthor {Stephen A. Butterfill}
\def \iemail{s.butterfill@warwick.ac.uk}
\date{}

\input{$HOME/Documents/submissions/preamble_steve_handout}

%for strikethrough
\usepackage[normalem]{ulem}

%itemize bullet should be dash
\renewcommand{\labelitemi}{$-$}

\begin{document}

\begin{multicols}{3}

\setlength\footnotesep{1em}

\bibpunct{}{}{,}{s}{}{,}  %use superscript TICS style bib

\bibliographystyle{newapa} %apalike

%\maketitle
%\tableofcontents






\begin{center}
{\Large
Mindreading \& Joint Action: Philosophical Tools}

Lecture 7: What Is Shared Agency?


ButterfillS@ceu.hu
\end{center}


How do activities involving shared agency differ from activities involving parallel agency only?

A \emph{joint} or \emph{collective action} is an exercise of shared agency.

\section{Shared agency requires shared intention?}

`I take a collective action to involve a collective intention.'  \citep%[p.\ 5]
{Gilbert:2006wr}

`The sine qua non of collaborative action is a joint goal [shared intention] and a joint commitment’ 
\citep%[p.\ 181]
{tomasello:2008origins}

`the key property of joint action lies in its internal component \ldots \ in the participants’ having a ``collective'' or ``shared'' intention.' \citep%[pp. 444-5]
{alonso_shared_2009}

`Shared intentionality is the foundation upon which joint action is built.' \citep%[p.\ 381]
{Carpenter:2009wq}

`it is precisely the meshing and sharing of psychological states \ldots \ that holds the key to understanding how humans have achieved their sophisticated and numerous forms of joint activity'
\citep%[p.\ 369]
{Call:2009fk}



\section{What is shared intention?}

The functional role of shared intentions is to: 
(i) coordinate activities; (ii) coordinate planning; and (iii) provide a framework to structure bargaining.\citep%[p.\ 99]
{Bratman:1993je}

For you and I to have a shared intention that we J it is sufficient that: `(1)(a) I intend that we J and (b) you intend that we J; (2) I intend that we J in accordance with and because of la, lb, and meshing subplans of la and lb; you intend that we J in accordance with and because of la, lb, and meshing subplans of la and lb; (3) 1 and 2 are common knowledge between us.'\citep%[View 4]
{Bratman:1993je}


`each agent does not just intend that the group perform the […] joint action. Rather, each agent intends as well that the group perform this joint action in accordance with subplans (of the intentions in favor of the joint action) that mesh'\citep%[p.\ 332]
{Bratman:1992mi}

`shared intentional agency consists, at bottom, in interconnected planning agency of the participants'\citep{Bratman:2011fk} %[p.\ 11]


\section{A challenge}
`the contribution of lower-level processes to social interaction has hardly been considered. This has led philosophers to postulate complex intentional structures that often seem to be beyond human cognitive ability in real-time social interactions.'\citep{Knoblich:2008hy} %[p.\ 2022]

\section{A conjecture}
The prior existence of capacities for shared agency partially explains how sophisticated forms of mindreading emerge in evolution or development (or both).

`[F]unctions traditionally considered hallmarks of individual cognition originated through the need to interact with others ...\
perception, action, and cognition are grounded in social interaction.'\citep{Knoblich:2006bn} %[p.\ 103]

`the unique aspects of human cognition ... were driven by, or even constituted by, social co-operation.'\citep{Moll:2007gu} %[p.\ 1]

`human cognitive abilities ... [are] built upon social interaction' 
%(*ref Sinigaglia and Sparaci 2008)





\section{Shared Agency without Shared Intention}



\subsection{First attempt}
A joint action is an action with two or more agents?\citep{ludwig_collective_2007}

\emph{Objection} 
`our primitive actions, the ones we do not by doing something else, ... these are all the actions there are.'\citep%[p.\ 59]
{Davidson:1971fz}


\subsection{Second attempt}
A joint action is an \sout{action} event grounded by two or more agents’ actions.

Two or more events \emph{overlap} just if any (perhaps improper) part of one of these events is a (perhaps improper) part of any of the other events.

\textbf{singular grounding} 
Event $D$ \emph{grounds} $E$, if: $D$and $E$ occur; 
$D$ is a (perhaps improper) part of $E$; and 
$D$ causes every event that is a proper part of $E$ but does not overlap $D$.

To be the \emph{agent of an event} is to be the agent of the action which grounds it.\citep%[p.\ 81]
{pietroski_actions_1998}


\textbf{plural grounding} 
Events $D_1$, ...\ $D_n$ \emph{ground} $E$, if: $D_1$, ...\ $D_n$ and $E$ occur; 
$D_1$, ...\ $D_n$ are each (perhaps improper) parts of $E$; and 
every event that is a proper part of $E$ but does not overlap  $D_1$, ...\ $D_n$ is caused by some or all of $D_1$, ...\ $D_n$.

For an individual to be \emph{among the agents of an event} is for there to be actions $a_1$, ...\ $a_n$ which ground this event where the individual is an agent of some (one or more) of these actions.



\subsection{Further attempts}

A joint action is an event grounded by two or more agents’ actions where these actions have a \sout{distributive} \sout{collective} shared goal.

A \emph{goal} is an outcome to which actions are, or might be, directed.  A \emph{goal-state} is an intention or other state of an agent linking an action to a goal to which it is directed.

\begin{comment}
\begin{center}
  \includegraphics[width=0.3\textwidth]{standard_story.png}
\emph{Figure}: The standard story for individual action.
\end{center}
\end{comment}

\textbf{Distributive goal}.  The \emph{distributive goal} of two or more actions is G: (a) each action is individually directed to G; and (b) it is possible that: all actions succeed relative to this outcome.

\textbf{Collective goal}.  The \emph{collective goal} of two or more actions is G:
(a) G is a distributive goal of the outcomes;
(b) the actions are coordinated; and 
(c) coordination of this type would normally  facilitate occurrences of outcomes of G's type


\textbf{Shared goal}.  The \emph{shared goal} of two or more agents' actions is G: (a) G is a collective goal of their actions; 
(b) each agent can identify each of the other agents in a way that doesn't depend on knowledge of the goal or	 actions directed to it;
(c) each agent most wants and expects each of the other agents to perform actions directed to G; and 
(d) each agent most wants and expects G to occur as a common effect of all their goal-directed actions, or to be partly constituted by all of their goal-directed actions.






\footnotesize 
\bibliography{$HOME/endnote/phd_biblio}

\end{multicols}

\end{document}