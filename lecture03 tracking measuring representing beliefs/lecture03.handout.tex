%!TEX TS-program = xelatex
%!TEX encoding = UTF-8 Unicode

\documentclass[11pt]{extarticle}
% extarticle is like article but can handle 8pt, 9pt, 10pt, 11pt, 12pt, 14pt, 17pt, and 20pt text

\def \ititle {Joint Action \& the Emergence of Mindreading}
\def \isubtitle {Lecture 2: Minimal Theory of Mind}
\def \iauthor {Stephen A. Butterfill and Ian Apperly}
\def \iemail{s.butterfill@warwick.ac.uk}
\date{}

\input{$HOME/Documents/submissions/preamble_steve_handout}


%itemize bullet should be dash
\renewcommand{\labelitemi}{$-$}

\begin{document}

\begin{multicols}{3}

\setlength\footnotesep{1em}

\bibpunct{}{}{,}{s}{}{,}  %use superscript TICS style bib

\bibliographystyle{newapa} %apalike

%\maketitle
%\tableofcontents






\begin{center}
{\Large
Mindreading \& Joint Action: Philosophical Tools}

Lecture 3: Tracking, Measuring and Representing Beliefs


ButterfillS@ceu.hu
\end{center}

\section{Question}
What could someone represent that would enable her to track, at least within limits, others' perceptions, knowledge states and beliefs including false beliefs? 




\section{Tracking mental states}

An \emph{ability to track} certain kinds of mental state is an ability that exists in part because exercising it brings benefits obtaining which depends on exploiting or influencing facts about mental states of those kinds.  

To \textit{track} a particular belief (or other mental state) is to exercise an ability to track those kind of mental states in such a way that, normally, one's thoughts or actions (or both) would carry information that a given subject has that belief (or other mental state).

***THINK: why not just define tracking in terms of carrying information (whether or not it brings any benefit)?  This would allow you to say that a machine can be built to track beliefs without the machine exploiting or influencing others' beliefs.


\section{Theory of mind cognition is hard}

Conceptually demanding:
\begin{itemize}\itemsep0pt
\item Acquisition takes several years\citep{Wimmer:1983dz,Wellman:2001lz}
\item Tied to the development of executive function\citep{Perner:1999yr,Sabbagh:2006ke} and language\citep{Astington2005ot}
\item Development facilitated by explicit training\citep{Slaughter:1996fv} and siblings\citep{Clements:2000nc,Hughes:2004zj}
\end{itemize}
%
Cognitively demanding: 
\begin{itemize}
\item Requires attention and working memory in fully competent adults\citep{Apperly:2008jv,McKinnon:2007rr}
\end{itemize}






\footnotesize 
\bibliography{$HOME/endnote/phd_biblio}

\end{multicols}

\end{document}