%!TEX TS-program = xelatex
%!TEX encoding = UTF-8 Unicode

\def \papersize {a4paper}
\documentclass[12pt,\papersize]{extarticle}
% extarticle is like article but can handle 8pt, 9pt, 10pt, 11pt, 12pt, 14pt, 17pt, and 20pt text

\def \ititle {Mindreading and Joint Action: Philosophical Tools}
\def \isubtitle {Notes for Lecture 1}
\def \iauthor {Stephen A. Butterfill}
\def \iemail{ButterfillS@ceu.hu}
%for anonymous submisison
%\def \iauthor {}
%\def \iemail{}
%\date{}

\input{$HOME/Documents/submissions/preamble_steve_paper2}

%avoid overhang
\tolerance=5000


\begin{document}

\setlength\footnotesep{1em}

\bibliographystyle{newapa} %apalike

%these two lines are for anonymous submission --- they remove author and date
%but don't forget to remove defs above as well --- otherwise it will be in the metadata
%\author{}
%\date{}


\maketitle
%\tableofcontents
%
%\begin{abstract}
%\noindent
%***
%\ 
%
%\noindent
%\textbf{Keywords:}
%Mindreading, Joint Action, Action, Belief, Intention, Representation, Mental State
%\end{abstract}
%



\section{Terminology}


\textit{Mindreading} is 
	the process of 
	identifying mental states and actions 
	as the mental states and actions of a particular subject 
	on the basis, ultimately, of bodily movements and their absence,
somewhat as reading is the process of identifying propositions on the basis of inscriptions \citep[p.\ 4]{Apperly:2010kx}.


A \textit{joint action} is an event with two or more agents \citep{ludwig_collective_2007}.


\section{Aim}
In this course I want to introduce a variety of new and established philosophical ideas related to mindreading and joint action.

In selecting what to cover, I was looking for ideas that (i) might usefully inform scientific research on mindreading or on joint action (or both) but (ii) have so far been neglected or misunderstood by cognitive scientists.

So the idea is to offer a  course in the philosophy of mind and action for scientists working on joint action or mindreading.


\section{Four Challenges}
There are three or four challenges that these philosophical ideas might be helpful for.

\subsection{First challenge: Decomposition}
The first challenge is to find ways of decomposing mindreading into components, in something like the way that actual reading can be decomposed into orthographic, lexical, syntactic, semantic and pragmatic components.

If someone said they had identified the reading mechanism---ordinary reading, not mindreading---or the reading module, it would be hard to make sense of this claim.  
Surely reading involves multiple mechanisms and processes.
Some might be modular, but it seems likely that others are not.
At the very least, it makes little sense to argue about such matters for reading taken as a whole.
Instead you have to look at components of reading, and ask about the mechanisms and processes these involve.

The first challenge, then, is find ways of decomposing mindreading into components.



\subsection{Second challenge: Kinds of mindreading}
The second, closely related, challenge is to find theoretically coherent and empirically motivated ways of distinguishing kinds of mindreading.

There are several primitive forms of reading that you can master without being a fully competent reader.
For example, there is:
% 
\begin{itemize}
	\item	the ability to recognise individual graphemes,
	\item the ability to segment sequences of graphemes into  words,
	\item	the ability to recover phonemes from graphemes and multigraphs and to blend these phonemes together into a word, 
	and
	\item the ability to map  a limited number of written words onto their spoken counterparts.
\end{itemize}
%	
Acquiring these primitive forms of reading is often a step towards becoming a fully competent reader.
And having these primitive forms of reading can be useful in their own right. 

The second challenge is to identify primitive forms of mindreading that are useful in their own right and that might be acquired on the way to becoming a fully competent mindreader.

While there are some existing proposals on what these might be, I think few people appreciate how difficult it is to find ways of distinguishing kinds of mindreading that are both theoretically coherent and empirically motivated.

One problem is that some researchers' views about mindreading seem to be based almost entirely on their own grasp of commonsense psychology.  
But an adult humans' commonsense psychology exhibits a kind of holism.
According to Donald Davidson,
%
\begin{quote}
`We are stuck with our two main ways of describing and explaining things, one which treats objects and events as mindless, and the other which treats objects and events as having propositional attitudes. I see no way of bridging the gap by introducing an intermediate vocabulary.' \citep[p.\ 697]{Davidson:2003bw}
\end{quote}
%
It may be tempting to dismiss this assertion on the grounds that we can readily describe an infant as excited by a clapping game, or as preferring one toy to another. 
On the surface at least, these descriptions seem neither to involve propositional attitudes nor to involve treating the infant as mindless; it seems there are non-propositional forms of excitement and preference which are nevertheless mental states.

Even so, Davidson is right that there is a genuine difficulty when it comes to understanding non-propositional counterparts of attitudes like belief and perception.  We cannot rely entirely on commonsense here because our commonsense concepts of perception, belief, intention and action exhibit a form of holism: we grasp them only if we understand their interdependent roles in reason explanations \citep{Davidson:1995lk,Davidson:1995nl}. 


\subsection{Third challege}
The third challenge concerns joint action.
It is to find ways of understanding how joint action might involve more than the coordinated and cooperative behaviours exhibited by some insects, while also not requiring sophisticated forms of mindreading at close to the limits of what human adults are capable of.

Why should we face this challenge?

\begin{quote}
`[F]unctions traditionally considered hallmarks of individual cognition originated through the need to interact with others ...\
perception, action, and cognition are grounded in social interaction.'\citep[p.\ 103]{Knoblich:2006bn}
\end{quote}

\begin{quote}
The `Vygotskian Intelligence Hypothesis': `the unique aspects of human cognition ... were driven by, or even constituted by, social co-operation.'\citep[p.\ 1]{Moll:2007gu}
\end{quote}


\section{That we do not understand what mindreading is}
Empirical questions about mindreading include:
\begin{itemize}
\item When in development does mindreading first occur?
\item What representations and processes make mindreading possible?
\item Is mindreading automatic?
\item Which animals are capable of mindreading?
\end{itemize}
%
Much progress has been made on these questions, and there is more still to make. 
I want to suggest that there is also an obstacle to progress.
The obstacle is that we don't adequately understand what mindreading is. 

Why think that we don't adequately understand what mindreading is? 
The strongest reason is this.
Some apparently puzzling patterns in findings about mindreading can be resolved by thinking carefully about what mindreading is. 
But we'll only be in a position to evaluate this claim right at the end, when we have reflected on what mindreading is.

There are, though, some hints that we might not adequately understand what mindreading is.
As we'll see, there are controversies concerning what mental states are, and what actions are.  
But when the topic is mindreading, these controversies are usually ignored and it is assumed that we all know what actions and mental states are. 
To better understand what mindreading is we will need to reflect on what actions and mental states are.

So my plan is to step back from empirical questions about mindreading and first focus on more narrowly philosophical issues about what mindreading is.
Having done this, we'll come back to the empirical questions about mindreading.  
The philosophical part is valuable to the extent that it supports progress with questions about when, how and where mindreading occurs.

But you might still be sceptical that philosophy is really needed.  Do we really not adequately understand what mindreading is?
You probably shouldn't take my word for it.
After all, not understanding things is what I do for a living.
So consider these questions 
[*more refined version in the plan for these lectures]:
\begin{itemize}
%
\item What evidence could in principle support the ascription of a particular belief to a given subject, and how does the evidence support the ascription?
%
\item \emph{Objectivity} Could a mindreader be able to identify beliefs despite not  understanding what it is for a belief to be true or false? 
%
\item \emph{Self-awareness}  Does being a mindreader entail being able, sometimes, to identify one's own mental states and actions? 
%
\item Could there be mindreaders who can identify intentions and knowledge states but not beliefs?
%
\item Does identifying an action necessarily involve representing an intention?
%
\end{itemize}
If we fully understood what mindreading was, we would be able to answer these questions in a principled way.
The fact that we can't shows that we don't fully understand what mindreading is.
And it suggests that we don't adequately understand it either.

To better understand what mindreading is we have to take a step back and ask what actions are and what mental states are.



%[NB: postpone empirical puzzles about mindreading until towards the end; use these to illustrate applications of the ideas.]




\section{What are mental states?}

mental state = 
	\\ \hspace*{10 mm} subject [e.g. Ayesha] 
	\\ \hspace*{10 mm} + 
	\\ \hspace*{10 mm} attitude [e.g. desire] 
	\\ \hspace*{10 mm} + 
	\\ \hspace*{10 mm} content [e.g. that Ayesha eats ice cream]

The subject is just an object.

Explain attitude and content using 2 x 2: 




\begin{table}[htbp]


\begin{center}
\footnotesize	%shrink for better spacing
\begin{tabular*}{1\textwidth}{@{\extracolsep{\fill}} l c *{3}{cc} } 

\toprule
& \multicolumn{3}{c}{attitude} 
\\ \cmidrule(r){2-4}
%
 & belief & desire & ...
%
\\ \midrule
%
Ayesha eats ice cream & 1 & 3 & ...
\\
Frederique writes poetry & 2 &  5 & ...
\\
... & ... & ... & ...
\\
%
\bottomrule
%
\end{tabular*}
\caption{Attitude versus content}
\end{center}	%careful -- position of this affects distance between table and caption(!)
\end{table}

The attitude is normally specified by its functional and normative roles, and these are usually explained in contrast with those of other attitudes.
E.g. What distinguishes believing from supposing?  These have related roles in guiding action.  Velleman (*ref) suggests that believing differs from supposing in aiming at truth.  We'll return to this idea later.

To specify the content we first need to identify something about its structure.
Mental states are usually thought of as having propositional contents.
But there is a variety of types of content that a mental state can have.
For instance, you can have an attitude towards a map-like structure, an image, an event-type, an object or a relation.

***examples (e.g. use navigation for attitudes towards maps?)

*Explain what propositions are (like numbers).  

*Also explain different types of propositions (Russellian, Fregean \&c)

*illustrate limits of different kinds of content (compare with different kinds of number)


\section{The origin of the attitudes}

Take an attitude like belief or desire.
Suppose someone offers a partial characterisation of the attitude.
For instance, 
suppose they say that belief aims at truth whereas desire aims at satisfaction.
What is this partial characterisation answerable to?
On what grounds should we accept or reject it?

We might treat claims about the attitudes as merely terminological stipulations, so that the only requirement is coherence.
%*rough(!):
This serves only to push back the question further.
What are we attempting to capture in characterising an attitude?  

Another possibility is to think of claims about the attitudes as answerable to ordinary thinking about mental states.
While I doubt we can escape ordinary thinking entirely, I think we should be cautious in appealing to it for two reasons.
One is that we don't actually know very much about how people ordinarily think about mental states.
The other is that ordinary thinking about mental states may not be right, or even consistent.

The approach to the attitudes I prefer is modelling.
This is going to take a while to explain but the idea is simple.
Decision theory provides us with a model capable of explaining, within limits, the preferences that agents manifest.
The model involves subjective probabilities and desirabilities, which roughly resemble belief and desire in some ways (\citealp[p.\ 59]{Jeffrey:1983oe}, \citealp{Davidson:1985qg}). 
So the model potentially provides two things.
One is a fairly precise characterisation of belief-like and desire-like attitudes.
The other is an explanation of when postulating them is justified.
Justification for postulating these states in a particular case depends on how well the model explains the agents' preferences.

\subsection{Actions, outcomes and conditions}
That was a bit abstract.
Let's get into details.
(What follows is based on \citealp{Jeffrey:1983oe}; it's not my own work.) 
I'm going to go very slowly at the start. 
This will be a bit painful if you're already familiar with decision theory, but it's worth it because we will use the basic ideas more than once.
(This will be important again in the context of joint action.)

Imagine we are deciding between two \emph{actions}, cycling to the seminar or catching the bus. 
Let's also suppose that we are only interested in two types of \emph{outcome} these actions could have, staying dry versus not staying dry and getting exercise versus not getting exercise.
Among the various possible outcomes, our preference ranking is:
%
\begin{enumerate}
\item getting exercise and staying dry
\item not getting exercise and staying dry
\item getting exercise and not staying dry
\end{enumerate}
%
Suppose we know that cycling will result in the third outcome, 3, whereas getting the bus will result in the second outcome, 2. 
Then we should get the bus.
In this situation, actions guarantee outcomes so how we act should depend just on which outcome we prefer.

Very often actions are not so simply linked to outcomes.
We often don't know whether we will get wet if we cycle.
Whether we get wet depends on further \emph{conditions}, such as whether it rains or whether there is flooding.
In general, which outcome occurs depends both on the actions we choose and on the conditions we encounter.


\begin{table}[htbp]
\begin{center}
\footnotesize	%shrink for better spacing
\begin{tabular*}{1\textwidth}{@{\extracolsep{\fill}} l c *{3}{cc} } 

\toprule

& \multicolumn{2}{c}{\emph{condition}} 
\\ 
\cmidrule(r){2-3}

 \emph{action} & no flooding & flooding
%
\\ \midrule
%
cycle & get exercise and stay dry & get exercise and get wet
\\
take bus & get no exercise and stay dry & get no exercise and stay dry 
\\
%
\bottomrule
%
\end{tabular*}
\caption{Outcomes depend on actions and conditions}
\end{center}	%careful -- position of this affects distance between table and caption(!)
\end{table}



Usually we are not certain about all the conditions relevant to a decision.
Instead we have to form a view about their probability.
For example, we might know that there is a fair chance of flooding without knowing outright that there is flooding.
This means we don't know for sure which outcome cycling will result in.
It might result in our most preferred outcome, getting exercise and staying dry; but it might also result in our least preferred outcome, getting exercise and getting wet.

For this reason, in deciding what to do we should ideally take into account both our preferences concerning the outcomes and  the probabilities of the conditions obtaining.
This could be done as follows.

We first consider the probabilities (see table \vref{table:probabilities}).
The top left cell represents the probability of no flooding if we choose to cycle.
%nb: if we choose to cycle, not *given that* we choose to cycle.  See http://plato.stanford.edu/entries/decision-causal/
Of course, the probability of flooding is independent of whether we cycle or take the bus.
But the approach is flexible enough to accommodate cases where our actions can influence the probability of the conditions ocurring.

\begin{table}[htbp]
\begin{center}
\footnotesize	%shrink for better spacing
\begin{tabular*}{1\textwidth}{@{\extracolsep{\fill}} l c *{3}{cc} } 

\toprule

& \multicolumn{2}{c}{\emph{condition}} 
\\ 
\cmidrule(r){2-3}

 \emph{action} & no flooding & flooding
%
\\ \midrule
%
cycle 
	& \begin{tabular}{c} 
		probability of 
		\\ no flooding if we cycle: 0.3 
	\end{tabular}
	& \begin{tabular}{c} 
		probability of 
		\\ flooding if we cycle:  0.7
	\end{tabular}
\\
take bus 
	& \begin{tabular}{c} 
		probability of 
		\\ no flooding if we get the bus: 0.3 
	\end{tabular}
	& \begin{tabular}{c} 
		probability of 
		\\ flooding if we get the bus: 0.7
	\end{tabular}
\\
%
\bottomrule
%
\end{tabular*}
\caption{Probabilities of four conditions obtaining}
\label{table:probabilities}
\end{center}	%careful -- position of this affects distance between table and caption(!)
\end{table}

We then assign weights to the outcomes that reflect how desirable they are in relation to each other (see table \vref{table:desirabilities}).

\begin{table}[htbp]
\begin{center}
\footnotesize	%shrink for better spacing
\begin{tabular*}{1\textwidth}{@{\extracolsep{\fill}} l c *{3}{cc} } 

\toprule

& \multicolumn{2}{c}{\emph{condition}} 
\\ 
\cmidrule(r){2-3}

 \emph{action} & no flooding & flooding
%
\\ \midrule
%
cycle 
	& \begin{tabular}{c} 
		[get exercise and stay dry] 
		\\ desirability of outcome: 3 
	\end{tabular}
	& \begin{tabular}{c} 
		[get exercise and get wet]
		\\ desirability of outcome:  -1
	\end{tabular}
\\
take bus 
	& \begin{tabular}{c} 
		[get no exercise and stay dry] 
		\\ desirability of outcome: 1 
	\end{tabular}
	& \begin{tabular}{c} 
		[get no exercise and stay dry] 
		\\ desirability of outcome:  1
	\end{tabular}
\\
%
\bottomrule
%
\end{tabular*}
\caption{Desirabilities of four outcomes}
\label{table:desirabilities}
\end{center}	%careful -- position of this affects distance between table and caption(!)
\end{table}


Finally we multiply the probability and desirability matrices (see table \vref{table:utilities}).
Adding the rows then gives expected utilities for each action.
The idea is that we should perform the act with the greatest expected utility.
In this case, cycling gets just 0.2 whereas taking the bus gets 1, so we should take the bus.

\begin{table}[htbp]
\begin{center}
\footnotesize	%shrink for better spacing
\begin{tabular*}{1\textwidth}{@{\extracolsep{\fill}} l c *{3}{cc} } 

\toprule

& \multicolumn{2}{c}{\emph{condition}} 
\\ 
\cmidrule(r){2-3}

 \emph{action} & no flooding & flooding
%
\\ \midrule
%
cycle & 0.3 [probability] $*$ 3 [desirability] = 0.9 & 0.7 $*$ -1 = -0.7
\\
take bus &   0.3 $*$ 1 = 0.3 &  0.7 $*$ 1 = 0.7
\\
%
\bottomrule
%
\end{tabular*}
\caption{Multiplying probabilities by desirabilities}
\label{table:utilities}
\end{center}	%careful -- position of this affects distance between table and caption(!)
\end{table}

I've gone very slowly over some familiar ideas because I want us to attend to the basics.
There are three basic elements: actions, conditions and outcomes.
Which outcome occurs depends on two things: the action chosen and the conditions that obtain.
Desirabilities attach to outcomes.
Probabilities attach to conditions.
Expected utilities attach to actions.
Expected utilities can be derived from desirabilities and probabilities.
One procedure for choosing an action is to compute expected utilities (in the way illustrated) and then perform the action with the highest expected utility.


\subsection{Reversing direction}
For simplicity I have so far been speaking as if we wanted to introduce a procedure for deciding how to act.
But of course that isn't our aim at all.
Our ultimate aim is to be able to justify  claims about attitudes---or at least to understand what sort of considerations might provide justification.
We still have a way to go.

So far we have seen how subjective probabilities and desirabilities determine expected utilities for actions.
If we knew the expected utilities of the actions available to an agent, we could make a prediction about how she will act---about whether she will cycle or take the bus, say.
But to work out an agent's expected utilities we would need to know the probabilities she assigns to relevant conditions and how desirable she finds the various outcomes.
How could we know this?
\textbf{What is the evidential basis for ascribing an agent subjective probabilities and desirabilities and how does the evidence support the ascriptions?}

Suppose we knew these things:
\begin{enumerate}
\item The agent will always perform an action with at least as much expected utility as any other actions available to her.
\item The agent has just two actions available to her: she can either cycle or get the bus.
\item The agent is getting the bus.
\end{enumerate}
Then we also know that the agent assigns the same, or higher, expected utility to getting the bus than to cycling.
So given some assumptions, an agent's actions reveal her expected utilities.

But this doesn't reveal much about the agent's preferences or probabilities.  
After all, expected utility is a function of both.
The agent might be getting the bus because she does not particularly desire exercise.
Or the agent might be getting the bus \emph{despite} particularly desiring exercise because she assigns a high probability to flooding (and a low desirability to staying dry).

On the face of it then, an agent's actions might reveal at most her expected utilities (given assumptions listed above) and leave us blind to her subjective probabilities and desirabilities.

If we knew the agent's subjective probabilities then, given some assumptions, we could work out her desirabilities from her expected utilities.
Someone who thinks flooding is likely but nevertheless cycles when she could have taken the bus must desire exercise more than staying dry. 




***HERE
*** intuitive explanation of Ramsey's criterion.



\subsection{Ramsey's insight}
Ramsey's criterion:
\begin{quote}
`Suppose that A and B are consequences [outcomes] between which the agent is not indifferent, and that N is an ethically neutral condition [i.e.\ the agent is indifferent between N and not N]. 
%strictly:  p 46: A con- dition is ethically neutral in relation to a particular agent and a particular consequence if the agent is indifferent between having that consequence when the condition holds and when it fails.
Then N has probability 1/2 if and only if the agent is indifferent between the following two gambles.
	\\ \hspace*{10 mm} B if N, A if not 
	\\ \hspace*{10 mm} A if N, B if not'
	\citep[p.\ 47]{Jeffrey:1983oe}
\end{quote}




 


\subsection{What is subjective desirability?  What is subjective probability?}
So far I have also not said what the desirabilities and probabilities are.
You may have some intuitions about what these represent.
In particular,  you may notice that they are intuitively related to desires and beliefs.
But strictly speaking so far we should treat as unresolved the issue of what these desirabilities and probabilities are.
Our aim is to use the desirabilities and probabilities to elucidate attitudes like belief and desire.
So it would clearly be a mistake to rely on those attitudes in saying what desirabilities and probabilities are.



\subsection{Towards a model}
We want  a model to explain actions.

Here's the idea in outline.
We want a method that will enable us to assign subjective desirabilities and subjective probabilities to an agent by observing some of their actions.
And we then want to be able to \emph{predict} their actions using the assigned subjective desirabilities and subjective probabilities.



\subsection{A gap}
***So far we have a theory of how attitudes relate to action but we do not have a theory about how they are acquired (a theory of belief fixation).
\begin{quote}
`A theory of mind needs a story about mental processes, not just a story about mental states. ... the logical behaviourism of Wittgenstein and Ryle had, as far as I can tell, no theory of thinking at all (except, maybe, the silly theory that thinking is talking to oneself). I do find that shocking. How could they have expected to get it right about belief and the like without getting it right about belief fixation and the like?' \citep[p.\ 9--10]{Fodor:1998ap}
\end{quote}
\begin{quote}
`modern philosophers ... have no theory of thought to speak of. I do think this is appalling; how can you seriously hope for a good account of belief if you have no account of belief fixation?' 
\citep[p.\ 147]{Fodor:1987rt}
\end{quote}




\small
\bibliography{$HOME/endnote/phd_biblio}

\end{document}