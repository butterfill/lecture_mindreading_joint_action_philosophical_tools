%!TEX TS-program = xelatex
%!TEX encoding = UTF-8 Unicode

\documentclass[11pt]{extarticle}
% extarticle is like article but can handle 8pt, 9pt, 10pt, 11pt, 12pt, 14pt, 17pt, and 20pt text

\def \ititle {Joint Action \& the Emergence of Mindreading}
\def \isubtitle {Lecture 2: Minimal Theory of Mind}
\def \iauthor {Stephen A. Butterfill and Ian Apperly}
\def \iemail{s.butterfill@warwick.ac.uk}
\date{}

\input{$HOME/Documents/submissions/preamble_steve_handout}


%itemize bullet should be dash
\renewcommand{\labelitemi}{$-$}

\begin{document}

\begin{multicols}{3}

\setlength\footnotesep{1em}

\bibpunct{}{}{,}{s}{}{,}  %use superscript TICS style bib

\bibliographystyle{newapa} %apalike

%\maketitle
%\tableofcontents






\begin{center}
{\Large
Mindreading \& Joint Action: Philosophical Tools}

Lecture 1: Introduction


ButterfillS@ceu.hu
\end{center}



\section{Terminology}
\textit{Mindreading} is 
	the process of 
	identifying mental states and actions 
	as the mental states and actions 	of a particular subject 
	on the basis, ultimately, of bodily movements and their absence,
somewhat as reading is the process of identifying propositions on the basis of inscriptions.\citep%[p.\ 4]
{Apperly:2010kx}

A \textit{joint action} is an event with two or more agents.\citep{ludwig_collective_2007}

`joint action can be regarded as any form of social interaction whereby two or more individuals coordinate their actions in space and time to bring about a change in the environment'\citep%[p.\ 70]
{Sebanz:2006yq}



\section{Quotes}
`We are stuck with our two main ways of describing and explaining things, one which treats objects and events as mindless, and the other which treats objects and events as having propositional attitudes. I see no way of bridging the gap by introducing an intermediate vocabulary.' \citep%[p.\ 697]
{Davidson:2003bw}


`[F]unctions traditionally considered hallmarks of individual cognition originated through the need to interact with others ...\
perception, action, and cognition are grounded in social interaction.'\citep%[p.\ 103]
{Knoblich:2006bn}

The `Vygotskian Intelligence Hypothesis': `the unique aspects of human cognition ... were driven by, or even constituted by, social co-operation.'\citep%[p.\ 1]
{Moll:2007gu}



\section{Infant false-belief tracking abilities}
One-year-old children predict actions of agents with false beliefs about the locations of objects\citep{Clements:1994cw,Onishi:2005hm,Southgate:2007js} and about the contents of containers,\citep{he:2011_false} 
taking into account verbal communication.\citep{Song:2008qo,scott:2012_verbal_fb}
They will also choose ways of helping\citep[]{Buttelmann:2009gy} and communicating\citep{Knudsen:2011fk,southgate:2010fb} with others depending on whether their beliefs are true or false.  
And in much the way that irrelevant facts about the contents of others’ beliefs modulate adult subjects’ response times, such facts also affect how long 7-month-old infants look at some stimuli.\citep[]{kovacs_social_2010}


\section{Three-year-olds fail false belief tasks}
Three-year-olds systematically fail to predict actions\citep{Wimmer:1983dz} and desires\citep{Astington:1991kk} based on false beliefs; they simililarly fail to retrodict beliefs\citep{Wimmer:1998kx} and to select arguments suitable for agents with false beliefs.\citep{Bartsch:2000es}
They fail some nonverbal false belief tasks;\citealp{Call:1999co,low:2010_preschoolers}
they fail whether the question concerns others' or their own (past) false beliefs;\citep{Gopnik:1991db}
and they fail whether they are interacting or observing.\citep{Chandler:1989qa}

\section{B-tasks}
By stipulation, B-tasks have these features:
\begin{itemize}
\item Children tend to pass them some time after their third birthday.
\item Abilities to pass these tasks has a protracted developmental course stretching over months if not years.
\item Success on these tasks is correlated with developments in executive function\citep[]{Perner:1999yr, Sabbagh:2006ke} and language.\citep[]{Astington2005ot}
\item Success on these tasks is facilitated by explicit training\citep[]{Slaughter:1996fv} and environmental factors such as siblings.\citep[]{Clements:2000nc, Hughes:2004zj}
\item Abilities to succeed on these tasks typically emerge from extensive participation in social interactions.\citealp{Hughes:2006fu}
\end{itemize}


The pattern of failure indicates a single developmental transition.\citep{Wellman:2001lz}

\section{Puzzle}
\begin{enumerate}
\item There are subjects who can pass A-tasks but cannot pass B-tasks.

\item These subjects’ success on A-tasks is explained by the fact that they can represent (false) beliefs

\item These subjects’  failure on B-tasks is explained by the fact that they cannot represent (false) beliefs
\end{enumerate}



\footnotesize 
\bibliography{$HOME/endnote/phd_biblio}

\end{multicols}

\end{document}